% =============================================================================
% MASTER THESIS DEFENSE PRESENTATION
% Intelligent Analysis of Automotive Sensor Faults Using
% Self-Supervised Learning and Real-Time Simulation
% Yahia Amir Yahia Gamal | TU Clausthal | 2025
% =============================================================================
% NOTE: This uses a plain Beamer theme. Replace with your university template
% by changing \usetheme and adding your background images.
% =============================================================================

\documentclass[aspectratio=169,12pt]{beamer}

% --- Use a clean theme (user will override with uni background) ---
\usetheme{default}
\usecolortheme{default}
\setbeamertemplate{navigation symbols}{}
\setbeamertemplate{footline}[frame number]
\setbeamerfont{frametitle}{size=\large,series=\bfseries}

\usepackage[T1]{fontenc}
\usepackage[utf8]{inputenc}
\usepackage{graphicx}
\graphicspath{{figures/}}
\usepackage{booktabs}
\usepackage{amsmath}
\usepackage{multirow}
\usepackage{tikz}
\usepackage{xcolor}
\usepackage{array}

% Custom colors
\definecolor{tucgreen}{RGB}{0,140,80}
\definecolor{tucblue}{RGB}{0,90,160}
\definecolor{tucred}{RGB}{180,30,30}
\definecolor{tucorange}{RGB}{220,130,0}
\definecolor{tucgray}{RGB}{100,100,100}
\definecolor{lightbg}{RGB}{240,245,250}

\setbeamercolor{frametitle}{fg=tucblue}
\setbeamercolor{title}{fg=tucblue}
\setbeamercolor{structure}{fg=tucblue}

\title{Intelligent Analysis of Automotive Sensor Faults\\Using Self-Supervised Learning and\\Real-Time Simulation}
\author{Yahia Amir Yahia Gamal}
\institute{Institute of Software and Systems Engineering\\TU Clausthal}
\date{2025}

\begin{document}

% =====================================================================
% SLIDE 1: TITLE
% =====================================================================
\begin{frame}[plain]
\vfill
\centering
{\LARGE\bfseries\color{tucblue} Intelligent Analysis of Automotive\\Sensor Faults Using Self-Supervised\\Learning and Real-Time Simulation}

\vspace{1.5em}
{\large Yahia Amir Yahia Gamal}

\vspace{0.8em}
{\normalsize Master Thesis Defense}\\[0.3em]
{\small Institute of Software and Systems Engineering\\
Faculty of Mathematics/Computer Science and Mechanical Engineering\\
TU Clausthal}

\vspace{1em}
{\small
\begin{tabular}{ll}
First Examiner: & apl. Prof. Dr. Christoph Knieke\\
Second Examiner: & Dr. Stefan Wittek\\
Supervisor: & Dr. Mohammad Abboush
\end{tabular}}

\vspace{0.5em}
{\footnotesize 2025}
\vfill
\end{frame}

% =====================================================================
% SLIDE 2: OUTLINE
% =====================================================================
\begin{frame}{Outline}
\begin{enumerate}
    \setlength{\itemsep}{0.8em}
    \item Motivation and Problem Statement
    \item Background: V-Model and Contrastive Learning
    \item Research Questions and Objectives
    \item Proposed Framework (3 Phases)
    \item Dataset: A2D2 and HIL
    \item SimCLR Training and 1D-CNN Encoder
    \item Fault Detection Results
    \item Binary Classification and ROC Analysis
    \item Computing Costs
    \item Conclusion and Future Work
\end{enumerate}
\end{frame}

% =====================================================================
% SLIDE 3: MOTIVATION
% =====================================================================
\begin{frame}{Motivation}
\begin{columns}[T]
\begin{column}{0.55\textwidth}
\textbf{The Problem:}
\begin{itemize}
    \setlength{\itemsep}{0.5em}
    \item Modern vehicles contain 100+ sensors
    \item Sensor faults compromise safety (ISO~26262)
    \item Traditional methods need \textbf{labeled fault data}
    \item Labeled data is expensive, incomplete, and cannot cover all failure modes
\end{itemize}

\vspace{1em}
\textbf{Our Solution:}
\begin{itemize}
    \setlength{\itemsep}{0.3em}
    \item Learn \textbf{only from healthy data}
    \item Self-supervised contrastive learning (SimCLR)
    \item Detect faults as \textit{deviations from normal}
    \item No labeled fault examples needed
\end{itemize}
\end{column}
\begin{column}{0.42\textwidth}
\includegraphics[width=\textwidth]{vmodel_diagram.png}
\end{column}
\end{columns}
\end{frame}

% =====================================================================
% SLIDE 4: RESEARCH QUESTIONS
% =====================================================================
\begin{frame}{Research Questions}
\begin{block}{\textbf{RQ1:} Effectiveness of SSL}
Can self-supervised contrastive learning on healthy driving data produce representations that effectively detect sensor faults in HIL simulation data?
\end{block}

\vspace{0.5em}

\begin{block}{\textbf{RQ2:} Threshold Trade-offs}
How does the detection threshold percentile affect the trade-off between precision and recall, and what is the optimal operating point?
\end{block}

\vspace{0.5em}

\begin{block}{\textbf{RQ3:} Fault Type Detectability}
Which fault types (gain, noise, stuck-at) and sensor locations (accelerator, speed) are most and least detectable, and why?
\end{block}
\end{frame}

% =====================================================================
% SLIDE 5: PROPOSED FRAMEWORK
% =====================================================================
\begin{frame}{Proposed Framework: Three-Phase Architecture}
\centering
\includegraphics[width=0.92\textwidth]{framework_diagram.png}
\end{frame}

% =====================================================================
% SLIDE 6: DATASETS
% =====================================================================
\begin{frame}{Datasets: A2D2 and HIL}
\begin{columns}[T]
\begin{column}{0.48\textwidth}
\textbf{A2D2 (Training)}
\begin{itemize}
    \setlength{\itemsep}{0.3em}
    \item Audi real-world driving data
    \item 3 recordings from Munich
    \item 219,064 samples (36.5 min @ 100\,Hz)
    \item Sensors: accelerator (\%), speed (km/h)
    \item Speed upsampled: 50\,Hz $\rightarrow$ 100\,Hz
    \item \textbf{Only healthy data} used for training
\end{itemize}
\end{column}
\begin{column}{0.48\textwidth}
\textbf{HIL (Testing)}
\begin{itemize}
    \setlength{\itemsep}{0.3em}
    \item dSPACE HIL simulator at TU Clausthal
    \item 1 healthy + 6 fault recordings
    \item Fault types: gain, noise, stuck-at
    \item Sensors: accelerator + speed
    \item 90\,s healthy used for calibration
    \item \textbf{Cross-domain transfer}: train on real $\rightarrow$ test on simulation
\end{itemize}
\end{column}
\end{columns}

\vspace{0.8em}
\centering
\includegraphics[width=0.75\textwidth]{dataset_fusion_comparison.png}
\end{frame}

% =====================================================================
% SLIDE 7: FAULT TYPES
% =====================================================================
\begin{frame}{Fault Injection Types}
\centering
\includegraphics[width=0.82\textwidth]{fault_types_comparison.png}

\vspace{0.3em}
\small
\begin{tabular}{lll}
\toprule
\textbf{Fault} & \textbf{Model} & \textbf{Physical Cause}\\
\midrule
Gain & $y(t) = \alpha \cdot x(t)$ & Calibration error, amplifier drift\\
Noise & $y(t) = x(t) + \eta(t),\ \eta \sim \mathcal{N}(0, \sigma^2)$ & EMI, loose connections\\
Stuck-at & $y(t) = c$ & Frozen sensor, hardware failure\\
\bottomrule
\end{tabular}
\end{frame}

% =====================================================================
% SLIDE 8: SIMCLR ARCHITECTURE
% =====================================================================
\begin{frame}{SimCLR: Contrastive Learning Framework}
\centering
\includegraphics[width=0.80\textwidth]{simclr_architecture.png}

\vspace{0.5em}
\textbf{NT-Xent Loss:}
$\displaystyle \ell_{i,j} = -\log \frac{\exp\!\big(\text{sim}(z_i, z_j)/\tau\big)}{\sum_{k \neq i} \exp\!\big(\text{sim}(z_i, z_k)/\tau\big)}$
\quad with $\tau = 0.5$, batch size = 128
\end{frame}

% =====================================================================
% SLIDE 9: ENCODER ARCHITECTURE
% =====================================================================
\begin{frame}{1D-CNN Encoder Architecture}
\centering
\includegraphics[width=0.85\textwidth]{encoder_architecture.png}

\vspace{0.5em}
\small
\begin{tabular}{lccccc}
\toprule
\textbf{Block} & \textbf{In$\rightarrow$Out Ch.} & \textbf{Kernel} & \textbf{Stride} & \textbf{Output} & \textbf{Params}\\
\midrule
Block 1 & 2 $\rightarrow$ 64 & 7 & 2 & $(B, 64, 50)$ & 960\\
Block 2 & 64 $\rightarrow$ 128 & 5 & 2 & $(B, 128, 12)$ & 41,216\\
Block 3 & 128 $\rightarrow$ 256 & 3 & 1 & $(B, 256, 6)$ & 99,072\\
\midrule
\multicolumn{5}{l}{+ BatchNorm + AdaptiveAvgPool} & \textbf{141,504}\\
\bottomrule
\end{tabular}
\end{frame}

% =====================================================================
% SLIDE 10: TRAINING RESULTS
% =====================================================================
\begin{frame}{Training Results}
\begin{columns}[T]
\begin{column}{0.55\textwidth}
\includegraphics[width=\textwidth]{part2_training_curves.png}
\end{column}
\begin{column}{0.42\textwidth}
\textbf{Training Summary:}
\begin{itemize}
    \setlength{\itemsep}{0.4em}
    \item 2,189 windows (200 samples each)
    \item 50 epochs, 17 batches/epoch
    \item Initial loss: 4.078
    \item Final loss: 3.791 (7\% reduction)
    \item Training time: \textbf{26.6\,s} on CPU
    \item Total params: 240,704
\end{itemize}

\vspace{0.5em}
\textbf{Augmentations:}
\begin{itemize}
    \item Gaussian jitter ($\sigma = 0.1$)
    \item Amplitude scaling ($[0.8, 1.2]$)
    \item Temporal masking (10\%)
\end{itemize}
\end{column}
\end{columns}
\end{frame}

% =====================================================================
% SLIDE 11: ANOMALY DETECTION METHOD
% =====================================================================
\begin{frame}{Anomaly Detection: How It Works}
\centering
\includegraphics[width=0.88\textwidth]{anomaly_detection_flow.png}

\vspace{0.8em}
\begin{columns}[T]
\begin{column}{0.48\textwidth}
\textbf{Calibration (Phase 2):}
\begin{enumerate}
    \item Pass 89 healthy HIL windows through frozen encoder
    \item Compute centroid $\mu_h = \frac{1}{N}\sum h_i$
    \item Compute similarity distribution
    \item Set thresholds at percentiles: 15th, 20th, 25th, 30th, 35th, 40th
\end{enumerate}
\end{column}
\begin{column}{0.48\textwidth}
\textbf{Detection (Phase 3):}
\begin{enumerate}
    \item Encode test window $\rightarrow$ embedding $h$
    \item Compute $\text{sim}(h, \mu_h) = \frac{h \cdot \mu_h}{\|h\|\|\mu_h\|}$
    \item If $\text{sim} < \tau$: \textcolor{tucred}{\textbf{Faulty}}
    \item If $\text{sim} \geq \tau$: \textcolor{tucgreen}{\textbf{Healthy}}
\end{enumerate}
\end{column}
\end{columns}
\end{frame}

% =====================================================================
% SLIDE 12: KEY RESULTS - MULTI-THRESHOLD
% =====================================================================
\begin{frame}{Results: Multi-Threshold Detection Performance}
\begin{columns}[T]
\begin{column}{0.48\textwidth}
\includegraphics[width=\textwidth]{part3_metrics_vs_threshold.png}
\end{column}
\begin{column}{0.48\textwidth}
\small
\renewcommand{\arraystretch}{1.2}
\begin{tabular}{lcccc}
\toprule
\textbf{Pctl.} & \textbf{$\tau$} & \textbf{Prec.} & \textbf{Rec.} & \textbf{F1}\\
\midrule
15th & 0.889 & 1.000 & 0.646 & 0.780\\
20th & 0.920 & 1.000 & 0.826 & 0.904\\
25th & 0.945 & 1.000 & 0.875 & 0.933\\
\rowcolor{lightbg}
30th & 0.967 & 1.000 & 0.913 & 0.954\\
\rowcolor{lightbg}
35th & 0.969 & 1.000 & 0.919 & 0.957\\
\rowcolor{yellow!20}
\textbf{40th} & \textbf{0.971} & \textbf{1.000} & \textbf{0.929} & \textbf{0.963}\\
\bottomrule
\end{tabular}

\vspace{0.5em}
\textbf{Key findings:}
\begin{itemize}
    \item Precision = \textbf{1.0 at all thresholds}
    \item Best F1 = \textbf{0.963} at 40th pctl.
    \item ROC-AUC = \textbf{0.852}
\end{itemize}
\end{column}
\end{columns}
\end{frame}

% =====================================================================
% SLIDE 13: PER-FAULT RESULTS
% =====================================================================
\begin{frame}{Results: Per-Fault Recall Analysis}
\begin{columns}[T]
\begin{column}{0.48\textwidth}
\includegraphics[width=\textwidth]{part3_recall_heatmap.png}
\end{column}
\begin{column}{0.48\textwidth}
\small
\textbf{At 40th percentile threshold:}

\vspace{0.5em}
\renewcommand{\arraystretch}{1.15}
\begin{tabular}{lcc}
\toprule
\textbf{Fault} & \textbf{Recall} & \textbf{F1}\\
\midrule
Acc gain & 90.8\% & 0.952\\
Acc noise & 90.7\% & 0.951\\
Acc stuck & 88.6\% & 0.939\\
\textbf{Speed gain} & \textbf{100\%} & \textbf{1.000}\\
Speed noise & 95.6\% & 0.977\\
Speed stuck & 91.7\% & 0.957\\
\bottomrule
\end{tabular}

\vspace{0.8em}
\textbf{Observations:}
\begin{itemize}
    \item Speed faults easier to detect (lower natural variability)
    \item Speed gain: \textbf{100\%} recall
    \item Acc stuck hardest (88.6\%)
\end{itemize}
\end{column}
\end{columns}
\end{frame}

% =====================================================================
% SLIDE 14: BINARY CLASSIFICATION + ROC
% =====================================================================
\begin{frame}{Results: Binary Classification and ROC}
\begin{columns}[T]
\begin{column}{0.48\textwidth}
\centering
\includegraphics[width=0.95\textwidth]{part3_confusion_matrices.png}
\end{column}
\begin{column}{0.48\textwidth}
\centering
\includegraphics[width=0.95\textwidth]{part3_similarity_distributions.png}

\vspace{0.3em}
\small
\textbf{Binary results at 40th pctl:}
\begin{itemize}
    \item Accuracy: \textbf{91.4\%}
    \item TP=1,775 | FP=36 | FN=136 | TN=53
    \item ROC-AUC = \textbf{0.852}
\end{itemize}
\end{column}
\end{columns}
\end{frame}

% =====================================================================
% SLIDE 15: COMPUTING COSTS
% =====================================================================
\begin{frame}{Computing Cost Analysis}
\begin{columns}[T]
\begin{column}{0.55\textwidth}
\renewcommand{\arraystretch}{1.3}
\small
\begin{tabular}{lrc}
\toprule
\textbf{Phase} & \textbf{Time (s)} & \textbf{\% Total}\\
\midrule
Part 1: A2D2 loading & 10.80 & 20.2\%\\
Part 2: SimCLR training & 26.56 & 49.7\%\\
\quad (50 epochs on CPU) & & \\
Part 3: Inference & 1.01 & 1.9\%\\
Part 3: Evaluation & 0.15 & 0.3\%\\
Part 3: Visualization & 4.52 & 8.4\%\\
\midrule
\textbf{Total pipeline} & \textbf{53.49} & \textbf{100\%}\\
\bottomrule
\end{tabular}
\end{column}
\begin{column}{0.42\textwidth}
\textbf{Key takeaways:}
\begin{itemize}
    \setlength{\itemsep}{0.5em}
    \item \textbf{Training}: 26.6\,s (one-time cost)
    \item \textbf{Inference}: 0.5\,ms per window
    \item \textbf{Total}: under 1 minute
    \item Suitable for real-time deployment
\end{itemize}

\vspace{0.8em}
\textbf{vs. Supervised approach:}\\
Ghannoum CNN-GRU: 23,000\,s (6.4\,h)\\
\textcolor{tucgreen}{\textbf{Our approach: 860$\times$ faster}}
\end{column}
\end{columns}
\end{frame}

% =====================================================================
% SLIDE 16: CONCLUSION
% =====================================================================
\begin{frame}{Conclusion}
\textbf{Answers to Research Questions:}

\vspace{0.5em}
\begin{description}
    \setlength{\itemsep}{0.5em}
    \item[\color{tucblue}RQ1:] SSL representations from healthy A2D2 data \textbf{successfully detect} sensor faults in HIL data. Cross-domain transfer works without labeled fault examples.

    \item[\color{tucblue}RQ2:] Precision = 1.0 across all thresholds. Recall increases from 64.6\% to 92.9\%. \textbf{Optimal: 40th percentile} (F1 = 0.963).

    \item[\color{tucblue}RQ3:] \textbf{Speed gain} faults are easiest (100\% recall). \textbf{Accelerator stuck-at} faults are hardest (88.6\%). Detectability correlates with physical severity.
\end{description}

\vspace{0.5em}
\textbf{Key Contributions:}
\begin{enumerate}
    \item First application of SimCLR to automotive sensor fault detection
    \item Cross-domain transfer: real driving $\rightarrow$ HIL simulation
    \item Comprehensive multi-threshold evaluation with 6 fault scenarios
    \item Complete pipeline in under 1 minute on CPU
\end{enumerate}
\end{frame}

% =====================================================================
% SLIDE 17: FUTURE WORK
% =====================================================================
\begin{frame}{Future Work}
\begin{columns}[T]
\begin{column}{0.48\textwidth}
\textbf{Short-term:}
\begin{itemize}
    \setlength{\itemsep}{0.5em}
    \item Add more sensor channels (RPM, steering, brake)
    \item Fault \textit{localization}: which sensor is faulty?
    \item Adaptive thresholding mechanisms
    \item Test with real vehicle fault data
\end{itemize}
\end{column}
\begin{column}{0.48\textwidth}
\textbf{Long-term:}
\begin{itemize}
    \setlength{\itemsep}{0.5em}
    \item Compare with TS2Vec, MoCo, BYOL
    \item Multi-scale temporal analysis
    \item Explainability (XAI) for fault diagnosis
    \item Edge deployment for real-time in-vehicle detection
    \item Integration with ISO~26262 ASIL framework
\end{itemize}
\end{column}
\end{columns}
\end{frame}

% =====================================================================
% SLIDE 18: THANK YOU
% =====================================================================
\begin{frame}[plain]
\vfill
\centering
{\LARGE\bfseries\color{tucblue} Thank You}

\vspace{1.5em}
{\Large Questions?}

\vspace{2em}
{\normalsize Yahia Amir Yahia Gamal}\\[0.3em]
{\small Institute of Software and Systems Engineering\\TU Clausthal}

\vspace{2em}
{\footnotesize
\begin{tabular}{ll}
Supervisor: & Dr. Mohammad Abboush\\
First Examiner: & apl. Prof. Dr. Christoph Knieke\\
Second Examiner: & Dr. Stefan Wittek
\end{tabular}}
\vfill
\end{frame}

\end{document}
