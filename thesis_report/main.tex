\documentclass[
	a4paper,
	12pt,
	parskip=half,
	]{scrbook}

\usepackage[
	left=30mm,
	right=40mm,
	top=25mm,
	bottom=30mm
	]{geometry}

\usepackage[T1]{fontenc}
\usepackage[utf8]{inputenc}
\usepackage[english]{babel}
\usepackage[autostyle]{csquotes}
\usepackage{lmodern}
\usepackage{microtype}

\usepackage[style=authoryear-ibid,backend=biber,doi=false,isbn=false]{biblatex}
\addbibresource{references.bib}
\DeclareDelimFormat{nameyeardelim}{\addcomma\space}

\usepackage{scrhack}
\usepackage{graphicx}
\graphicspath{{figures/}}
\usepackage{acronym}
\let\oldacsu\acsu\renewcommand{\acsu}[1]{\oldacsu{#1}\label{acro:#1}}

\usepackage[onehalfspacing]{setspace}

% Colors
\usepackage{xcolor}
\definecolor{TUCgreen}{rgb}{0,0.55,0.31}
\definecolor{TUCgrey1}{rgb}{0.5,0.5,0.5}
\definecolor{TUCgrey2}{rgb}{0.9,0.9,0.9}
\definecolor{TUCred}{rgb}{0.55,0.11,0}

% Additional packages
\usepackage{subcaption}
\usepackage{nicefrac}
\usepackage{amsmath}
\usepackage{amsfonts}
\usepackage{amssymb}
\usepackage{bbm}            % for \mathbbm{1} indicator function
\usepackage{algorithm,algpseudocode}
\usepackage{float}
\usepackage{booktabs}
\usepackage{multirow}
\usepackage{longtable}
\usepackage{array}
\usepackage{tabularx}
\usepackage{listings}
\usepackage{url}

\usepackage{hyperref}
\def\MakeUppercaseUnsupportedInPdfStrings{\scshape}

% Code listing style
\lstset{
  language=Python,
  basicstyle=\ttfamily\small,
  keywordstyle=\color{blue},
  commentstyle=\color{TUCgrey1},
  stringstyle=\color{TUCred},
  breaklines=true,
  frame=single,
  numbers=left,
  numberstyle=\tiny\color{TUCgrey1},
  captionpos=b
}

% ============================================================
% Title page data
% ============================================================
\title{Master Thesis}
\subtitle{Intelligent Analysis of Automotive Sensor Faults\\Using Self-Supervised Learning and\\Real-Time Simulation}
\author{Yahia Amir Yahia Gamal}
\newcommand*{\studiengang}{MSc Informatik}
\newcommand*{\matrikelnummer}{XXXXXX}% <-- Fill in your Matrikelnummer
\newcommand*{\erstgutachter}{apl. Prof. Dr. Christoph Knieke}
\newcommand*{\zweitgutachter}{Dr. Stefan Wittek}
\newcommand*{\betreuer}{Dr. Mohammad Abboush}
\newcommand*{\institut}{Institute of Software and Systems Engineering}
\newcommand*{\abteilung}{Department of Software Engineering}
\newcommand*{\fakultaet}{Faculty of Mathematics/Computer Science and Mechanical Engineering}
\date{2025}

% Title page definition (identical to TU Clausthal template)
\makeatletter
\def\maketitle{%
	\newgeometry{margin=2cm}
	\pagenumbering{alph}
	\thispagestyle{empty}
	\includegraphics[width=130mm]{Logo-TUC-DE-RGB}%
	\vspace{20mm}%
	{\raggedright
		\begin{center}
			{\huge \textsc \@title}\vspace{-4ex}
			{\Huge \bfseries \singlespacing \textsf \@subtitle\par}\vspace{2ex}
			{\Large \@author}\\[4ex]
			{\large \fakultaet}\\
			{\large \institut}\\
			{\large \abteilung}\\
			\vfill
			\begin{tabular}{ll}
				Matriculation Number & \matrikelnummer \\
				Program              & \studiengang \\[2ex]
				First Examiner       & \erstgutachter \\
				Second Examiner      & \zweitgutachter \\
				Supervisor           & \betreuer \\[2ex]
				Submission Date      & \@date
			\end{tabular}
		\end{center}
	}
	\restoregeometry
    \clearpage
	\thispagestyle{empty}
	\if@twoside
        \null\newpage
    \fi
    \pagenumbering{roman}
}
\makeatother

% ============================================================
\begin{document}

    \maketitle

    % ---- Declaration of Authorship ----
    \addchap{Declaration of Authorship}

    I have read and understood the guidelines of the Clausthal University of Technology.
    I confirm that I have prepared this master thesis independently by myself. Any
    information taken from other sources and being reproduced in this thesis is clearly
    referenced.

    In terms of the general examination regulations, this work has not yet been
    submitted to any other examination division.

    I hereby agree that my master thesis may be exhibited in the institute's or
    university library and kept for inspection.

    \vspace{4em}
    \noindent Clausthal-Zellerfeld, \hfill Yahia Amir Yahia Gamal\\
    \textit{Location, Date} \hfill \textit{Signature}

    % ---- Abstract ----
    \addchap{Abstract}

    The evolution of Automotive Software Systems (ASSs) has led to growing complexity in sensor networks, increasing the risk of undetected faults that can compromise vehicle safety. Traditional fault detection methods depend on supervised learning with large labeled datasets, which are expensive to obtain and inherently incomplete---they cannot cover every possible failure mode.

    This thesis proposes a self-supervised contrastive learning framework for automotive sensor fault detection that requires no labeled fault data. A 1D Convolutional Neural Network encoder is trained using the SimCLR framework on healthy driving data from the Audi Autonomous Driving Dataset (A2D2), learning to map time-series sensor windows into a 256-dimensional embedding space. Anomalies are detected by measuring the cosine similarity between test embeddings and a healthy reference centroid computed from a small calibration set of Hardware-in-the-Loop (HIL) data (90~seconds).

    The framework is evaluated on six fault scenarios---gain, noise, and stuck-at faults injected into accelerator pedal and vehicle speed sensors---across six detection threshold percentiles (15th through 40th). Comprehensive metrics including precision, recall, F1-score, accuracy, confusion matrices, and ROC curves are reported for all experiments, alongside computing cost measurements for training and inference.

    Results demonstrate that self-supervised representations learned from real-world driving data transfer effectively to HIL simulation data for fault detection, without any labeled fault examples. The multi-threshold analysis quantifies the precision--recall trade-off, providing practical guidance for threshold selection in safety-critical applications.

    \vspace{1em}
    \noindent\textbf{Keywords:} Self-Supervised Learning, Contrastive Learning, SimCLR, Automotive Sensor Fault Detection, Hardware-in-the-Loop, Anomaly Detection, 1D-CNN, Transfer Learning

    % ---- Zusammenfassung ----
    \addchap{Zusammenfassung}

    Die zunehmende Komplexit\"at moderner Fahrzeugsoftwaresysteme erh\"oht das Risiko unerkannter Sensorfehler, die die Fahrzeugsicherheit gef\"ahrden k\"onnen. Herk\"ommliche Fehlererkennungsmethoden basieren auf \"uberwachtem Lernen mit gro\ss{}en gelabelten Datens\"atzen, deren Beschaffung kostenintensiv und inhärent unvollst\"andig ist.

    Diese Arbeit schl\"agt ein selbst\"uberwachtes kontrastives Lernverfahren zur Fehlererkennung bei Fahrzeugsensoren vor, das keine gelabelten Fehlerdaten ben\"otigt. Ein 1D-CNN-Encoder wird mittels des SimCLR-Frameworks ausschlie\ss{}lich auf gesunden Fahrdaten des Audi Autonomous Driving Dataset (A2D2) trainiert. Anomalien werden durch Kosinusähnlichkeit der Einbettungen zu einem gesunden Referenz-Zentroid erkannt, der aus einer kurzen Kalibrierungsaufnahme (90~Sekunden) von Hardware-in-the-Loop-Daten berechnet wird.

    Das Framework wird an sechs Fehlerszenarien evaluiert---Verst\"arkungsfehler, Rauschfehler und Stuck-at-Fehler in Gaspedalstellung und Fahrzeuggeschwindigkeit---\"uber sechs Schwellenwert-Perzentile (15.\ bis 40.). Pr\"azision, Recall, F1-Score, Genauigkeit, Konfusionsmatrizen und ROC-Kurven werden f\"ur alle Experimente berichtet, zusammen mit den Berechnungskosten f\"ur Training und Inferenz.

    % ---- Table of Contents ----
    \tableofcontents
    \addcontentsline{toc}{chapter}{Contents}

    % ---- List of Acronyms ----
    \addchap{List of Acronyms}
    \begin{acronym}[XXXXX]
    	\acro{A2D2}{Audi Autonomous Driving Dataset}
    	\acro{ADAS}{Advanced Driver Assistance Systems}
    	\acro{AI}{Artificial Intelligence}
    	\acro{ASIL}{Automotive Safety Integrity Level}
    	\acro{ASS}{Automotive Software System}
    	\acro{AUC}{Area Under the Curve}
    	\acro{BYOL}{Bootstrap Your Own Latent}
    	\acro{CAN}{Controller Area Network}
    	\acro{CNN}{Convolutional Neural Network}
    	\acro{CPC}{Contrastive Predictive Coding}
    	\acro{DL}{Deep Learning}
    	\acro{ECU}{Electronic Control Unit}
    	\acro{EMI}{Electromagnetic Interference}
    	\acro{FN}{False Negative}
    	\acro{FP}{False Positive}
    	\acro{GRU}{Gated Recurrent Unit}
    	\acro{HIL}{Hardware-in-the-Loop}
    	\acro{MBD}{Model-Based Development}
    	\acro{MIL}{Model-in-the-Loop}
    	\acro{ML}{Machine Learning}
    	\acro{MoCo}{Momentum Contrast}
    	\acro{NT-Xent}{Normalized Temperature-scaled Cross-Entropy}
    	\acro{PIL}{Processor-in-the-Loop}
    	\acro{ROC}{Receiver Operating Characteristic}
    	\acro{SIL}{Software-in-the-Loop}
    	\acro{SSL}{Self-Supervised Learning}
    	\acro{TN}{True Negative}
    	\acro{TP}{True Positive}
    	\acro{VIL}{Vehicle-in-the-Loop}
    	\acro{XAI}{Explainable Artificial Intelligence}
    \end{acronym}

    % ---- List of Figures ----
    \listoffigures
    \addcontentsline{toc}{chapter}{List of Figures}

    % ---- List of Tables ----
    \listoftables
    \addcontentsline{toc}{chapter}{List of Tables}

    % ============================================================
    % MAIN CHAPTERS
    % ============================================================
    \chapter{Introduction}\label{ch:introduction}
\pagenumbering{arabic}

\section{Context and Motivation}\label{sec:context}

Modern vehicles rely on a growing network of Electronic Control Units (ECUs) that continuously process data from dozens of sensors to govern safety-critical functions such as engine management, braking, and steering \parencite{abboush2022intelligent}. A single mid-range passenger car today can contain more than 70~ECUs communicating over Controller Area Network (CAN) buses, each dependent on accurate sensor readings to maintain safe operation. When a sensor produces faulty measurements---whether due to electrical interference, mechanical degradation, or component failure---the downstream control logic may issue incorrect commands, potentially leading to hazardous situations for occupants and other road users.

The ISO~26262 standard for functional safety of road vehicles mandates rigorous testing and validation of these systems, including Hardware-in-the-Loop (HIL) simulation and real-world test drives \parencite{iso26262_2018}. During such validation campaigns, vast quantities of time-series sensor data are recorded, often exceeding several gigabytes per test session. Manual inspection of this data by domain experts is not only time-consuming but also prone to human oversight, particularly for subtle or intermittent faults that may not produce immediately obvious signal distortions.

Artificial intelligence offers a path toward automated fault detection. Supervised deep learning methods, including Convolutional Neural Networks (CNNs) combined with Gated Recurrent Units (GRUs), have demonstrated strong performance in classifying fault types and locations within automotive systems \parencite{ghannoum2025explainable, abboush2022intelligent}. However, these approaches depend on large, labeled datasets that include examples of every fault type the model is expected to recognize. Collecting such labeled fault data requires deliberate fault injection campaigns and manual annotation---a process that is expensive, time-intensive, and inherently incomplete, since it cannot cover every conceivable fault mode that may arise in practice.

Self-supervised learning (SSL) offers a fundamentally different paradigm. Rather than learning from labeled examples, SSL methods extract supervisory signals from the structure of the data itself \parencite{jaiswal2021survey}. In the context of fault detection, this means a model can learn what \emph{normal} sensor behavior looks like by training exclusively on healthy data, which is abundantly available from routine test drives. Any subsequent deviation from the learned normal patterns can then be flagged as a potential anomaly, without ever having seen a single labeled fault example during training.

Among SSL approaches, contrastive learning has emerged as particularly effective for representation learning \parencite{chen2020simple}. The core principle is to train an encoder network such that similar inputs (augmented views of the same data sample) produce similar representations, while dissimilar inputs produce distant representations. SimCLR \parencite{chen2020simple}, one of the most widely adopted contrastive frameworks, has demonstrated state-of-the-art results across diverse domains, yet its application to automotive sensor fault detection remains largely unexplored.

This thesis investigates whether contrastive self-supervised learning, trained solely on healthy driving data from the Audi Autonomous Driving Dataset (A2D2) \parencite{a2d2_2020}, can effectively detect sensor faults injected through a HIL simulation environment. The work addresses a practical scenario: a model trained on real-world driving data from one vehicle is deployed to detect faults in a different data acquisition system (HIL test bench), testing the cross-domain generalization capability of the learned representations.

\section{Research Questions}\label{sec:rqs}

This thesis addresses the following research questions:

\begin{itemize}
    \item \textbf{RQ1:} Can a contrastive self-supervised learning model, trained exclusively on healthy sensor data from real-world driving recordings, reliably detect sensor faults in HIL simulation data without any labeled fault examples?

    \item \textbf{RQ2:} How does the selection of the anomaly detection threshold affect the trade-off between detection sensitivity (recall) and false alarm rate (precision), and what threshold range is most suitable for safety-critical automotive applications?

    \item \textbf{RQ3:} Which types of sensor faults (gain, noise, stuck-at) are most and least detectable by the proposed approach, and what physical factors explain the observed differences?
\end{itemize}

\section{Related Work}\label{sec:intro_related}

A detailed review of relevant literature is provided in Chapter~\ref{ch:related_work}. In brief, prior work on automotive fault detection has predominantly relied on supervised learning with labeled fault data \parencite{abboush2022intelligent, ghannoum2025explainable}. Self-supervised methods have been applied to fault diagnosis in rotating machinery (bearings, motors) \parencite{wang2023self, ding2022self, li2023contrastive}, but their adoption for automotive sensor systems remains limited. Furthermore, most existing SSL-based approaches still require a small number of labeled fault examples for fine-tuning, whereas the approach proposed here operates in a fully unsupervised fashion after pretraining.

\section{Solution}\label{sec:solution}

The proposed framework operates in three distinct phases:

\begin{enumerate}
    \item \textbf{Self-supervised pretraining.} A one-dimensional CNN encoder is trained on the SimCLR contrastive learning objective using healthy driving data from the A2D2 dataset. The encoder learns to map time-series sensor windows into a compact embedding space where semantically similar windows are placed close together. Three domain-specific augmentation strategies---Gaussian jittering, amplitude scaling, and temporal masking---generate the positive pairs required for contrastive training.

    \item \textbf{Anomaly detection calibration.} A small quantity of healthy HIL data (approximately 90~seconds) is passed through the frozen encoder to establish a reference distribution in the embedding space. The centroid of the healthy embeddings and the distribution of cosine similarities to that centroid are computed, and detection thresholds are set at multiple percentiles of this distribution.

    \item \textbf{Fault detection and evaluation.} HIL data containing injected faults (gain, noise, and stuck-at types across accelerator pedal and vehicle speed sensors) is evaluated against the calibrated thresholds. Performance is assessed using precision, recall, F1-score, accuracy, confusion matrices, and Receiver Operating Characteristic (ROC) curves across multiple threshold settings.
\end{enumerate}

\section{Objective}\label{sec:objective}

The primary objective of this thesis is to develop, implement, and evaluate a self-supervised contrastive learning framework for automotive sensor fault detection that satisfies the following requirements:

\begin{enumerate}
    \item Requires no labeled fault data during any phase of training or calibration.
    \item Achieves reliable detection of gain, noise, and stuck-at sensor faults in HIL simulation data.
    \item Provides a systematic multi-threshold analysis (15th through 40th percentile) to quantify the sensitivity--specificity trade-off.
    \item Demonstrates cross-domain transfer from real-world A2D2 data to HIL data.
    \item Reports all evaluation metrics mandated for safety-critical systems: precision, recall, F1-score, accuracy, confusion matrices, and ROC curves with AUC values.
    \item Documents computational costs for training and inference to assess deployment feasibility.
\end{enumerate}

\section{Structure}\label{sec:structure}

The remainder of this thesis is organized as follows:

\begin{itemize}
    \item \textbf{Chapter~\ref{ch:background}: Background.} Introduces the theoretical foundations, including model-based development and testing phases in the automotive V-model, the real test drive validation process, fault injection approaches, contrastive self-supervised learning with detailed mathematical formulations, and anomaly detection methods.

    \item \textbf{Chapter~\ref{ch:related_work}: Related Work.} Reviews current research in self-supervised learning for fault detection and diagnosis, contrastive learning frameworks, anomaly detection with learned representations, and deep learning for automotive fault detection. Research gaps addressed by this thesis are identified.

    \item \textbf{Chapter~\ref{ch:methodology}: Methodology, Implementation, and Results.} Presents the complete system architecture, followed by detailed descriptions of each component: data preprocessing (A2D2 loading, normalization, windowing), dataset fusion with HIL data, SimCLR training implementation, and anomaly detection with multi-threshold evaluation. Intermediate results and visualizations are included throughout.

    \item \textbf{Chapter~\ref{ch:results}: Results and Discussion.} Reports the full evaluation results, including per-fault detection performance across six thresholds, binary classification results (healthy vs.\ faulty), ROC analysis, per-fault-type analysis, and computing cost measurements. A detailed discussion interprets the findings in the context of practical automotive deployment.

    \item \textbf{Chapter~\ref{ch:conclusion}: Conclusion and Future Work.} Summarizes the contributions, answers the research questions, and outlines directions for future research.
\end{itemize}

    \chapter{Background}\label{ch:background}

This chapter introduces the theoretical foundations required to understand the proposed fault detection framework. It covers the automotive development and validation process, fault injection techniques used in HIL testing, the fundamentals of deep learning for time-series data, and---most critically---a detailed treatment of contrastive self-supervised learning with full mathematical formulations.

\section{Model-Based Development and Testing Phases}\label{sec:mbd}

Model-Based Development (MBD) has become the standard methodology in the automotive industry for designing and validating embedded software systems. MBD uses mathematical and simulation models to facilitate early design validation and verification, following the V-model development lifecycle \parencite{abboush2022intelligent}. The left side of the V-model represents the design and development stages, while the right side represents the corresponding verification and validation stages, progressing from abstract models to physical hardware.

\subsection{Model-in-the-Loop (MIL)}\label{sec:mil}

Model-in-the-Loop testing validates the system design within a purely virtual environment. Both the controller and the plant are represented as mathematical models, typically developed in tools such as MATLAB/Simulink. MIL enables early verification of algorithmic correctness, control strategy logic, and functional behavior without requiring any hardware components. Because MIL testing is entirely software-based, it offers the fastest iteration cycles and the lowest cost among all V-model testing phases.

\subsection{Software-in-the-Loop (SIL)}\label{sec:sil}

In SIL testing, the controller model is replaced by the actual generated source code, which runs on the host development computer against the simulated plant model. SIL testing verifies that the auto-generated code behaves identically to the original controller model, identifying discrepancies that may arise from code generation, numerical precision, or implementation-specific behavior.

\subsection{Processor-in-the-Loop (PIL)}\label{sec:pil}

PIL testing executes the controller software on the target microprocessor or ECU hardware, while the plant model still runs in simulation. This phase captures processor-specific effects such as fixed-point arithmetic, limited memory, interrupt handling, and execution timing that are invisible during SIL testing.

\subsection{Hardware-in-the-Loop (HIL)}\label{sec:hil}

HIL testing represents a critical validation stage where the complete ECU hardware, including its physical interfaces, runs against a real-time simulation of the vehicle plant model \parencite{dspace2023hil}. The HIL simulator generates realistic electrical signals that mimic the outputs of real sensors (e.g., accelerometer readings, wheel speed pulses, pedal position voltages) and receives the ECU's actuator commands in return. This closed-loop configuration allows validation of the entire hardware-software stack under conditions that closely approximate real vehicle operation, but in a safe, repeatable laboratory environment.

HIL testing is particularly valuable for fault injection studies because faults can be introduced systematically at precisely defined times, locations, and magnitudes---something that would be impractical or dangerous to do in a real vehicle. The dSPACE HIL platform, used in the research group at TU~Clausthal, provides Configuration Desk for system setup and Control Desk for real-time monitoring and data logging during test execution \parencite{abboush2022intelligent}.

\subsection{Vehicle-in-the-Loop (VIL)}\label{sec:vil}

VIL testing places a physical vehicle---or a high-fidelity driving simulator---within a virtual environment that generates realistic driving scenarios. The vehicle's control systems respond to simulated road conditions, traffic, and weather, while the virtual environment adapts in real time to the vehicle's actions. VIL bridges the gap between laboratory HIL testing and real-world test drives.

\section{The Real Test Drive Validation Process}\label{sec:real_test_drive}

The real test drive constitutes the final validation stage for automotive systems before production release. During these drives, the vehicle operates under actual road conditions while onboard data acquisition systems continuously record sensor measurements at high sampling rates. This section describes the process as it relates to the data used in this thesis.

\subsection{Test Drive Data Acquisition}\label{sec:data_acquisition}

During a real test drive, data acquisition systems record signals from all relevant sensors and ECUs over the vehicle's CAN bus. Typical recorded signals include:

\begin{itemize}
    \item Accelerator pedal position (\%), reflecting the driver's throttle input
    \item Vehicle speed (km/h), measured by wheel speed sensors
    \item Engine speed (RPM), from the crankshaft position sensor
    \item Steering wheel angle (degrees), from the steering column sensor
    \item Brake pedal position (\%), indicating braking intensity
    \item Lateral and longitudinal acceleration (m/s\textsuperscript{2}), from inertial measurement units
\end{itemize}

Sampling rates vary by sensor type. In the Audi A2D2 dataset used in this thesis, the accelerator pedal signal is recorded at approximately 100~Hz, while the vehicle speed signal is recorded at approximately 50~Hz. These different rates reflect the physical characteristics of each measurement: the accelerator pedal position can change rapidly with the driver's foot movement, while vehicle speed changes more gradually due to the vehicle's inertia.

\subsection{Data Volume and Analysis Challenges}\label{sec:data_challenges}

A single test drive of 30~minutes at 100~Hz sampling generates over 180,000 data points per sensor channel. With multiple sensors recorded simultaneously, the total data volume per test session can reach several gigabytes. Manual expert analysis of this volume is impractical, motivating the development of automated analysis methods.

The key challenge is that faults in real-world data are rare, unlabeled, and diverse. A sensor may produce subtly incorrect readings for only a few seconds within hours of normal operation. Traditional threshold-based monitoring catches only the most severe deviations, while intermittent or slowly drifting faults may go undetected. This motivates the self-supervised approach developed in this thesis, which learns a comprehensive model of normal behavior and can identify even subtle deviations.

\subsection{The A2D2 Dataset}\label{sec:a2d2_background}

The Audi Autonomous Driving Dataset (A2D2) \parencite{a2d2_2020} is a large-scale public dataset recorded by Audi AG for research in autonomous driving. It contains sensor data from multiple test drives conducted in three German cities: Gaimersheim, Ingolstadt, and Munich. The dataset includes camera images, LiDAR point clouds, and CAN bus signals recorded during real-world driving.

For the purpose of this thesis, only the CAN bus signals are used, specifically the accelerator pedal position and vehicle speed. These signals provide a representative pair of automotive sensors with a clear physical relationship: the accelerator pedal is a driver input, and the vehicle speed is the corresponding system output. The data is stored in JSON format with microsecond-resolution timestamps, enabling precise sampling rate analysis and signal alignment.

\section{Fault Injection Approaches}\label{sec:fault_injection}

Fault injection is a validation technique used to assess how a system behaves in the presence of faults \parencite{abboush2022intelligent}. ISO~26262 recommends fault injection at various stages of the V-model to verify that safety mechanisms correctly detect and mitigate faults. A fault injection scenario is characterized by three attributes: \emph{when} the fault occurs (fault time), \emph{where} it occurs (fault location), and \emph{what type} of fault is introduced (fault type).

\subsection{Fault Types}\label{sec:fault_types}

Table~\ref{tab:fault_types} summarizes the three fault types used in this thesis, along with their mathematical representations and typical physical causes.

\begin{table}[H]
\centering
\caption{Mathematical representations of signal fault types used in this thesis.}
\label{tab:fault_types}
\renewcommand{\arraystretch}{1.4}
\begin{tabular}{p{2.8cm} p{5.5cm} p{4.5cm}}
\toprule
\textbf{Fault Type} & \textbf{Mathematical Representation} & \textbf{Physical Cause} \\
\midrule
Healthy signal & $y(t) = x(t)$ & Normal operation \\
Gain fault & $y(t) = \alpha \cdot x(t), \quad \alpha \neq 1$ & Calibration error, amplifier drift \\
Noise fault & $y(t) = x(t) + \eta(t), \quad \eta \sim \mathcal{N}(0,\sigma^2)$ & EMI, sensor degradation \\
Stuck-at fault & $y(t) = c \quad \forall\, t \geq t_f$ & Broken wiring, component failure \\
\bottomrule
\end{tabular}
\end{table}

\begin{figure}[H]
\centering
\includegraphics[width=0.95\textwidth]{figures/fault_types_comparison.png}
\caption{Comparison of a healthy sensor signal with three fault types: gain, noise, and stuck-at.}
\label{fig:fault_types_visual}
\end{figure}

\textbf{Gain fault.} The sensor output is scaled by an incorrect factor $\alpha$. When $\alpha > 1$, the signal is amplified; when $\alpha < 1$, it is attenuated. This fault typically arises from calibration errors in the signal conditioning circuit or from amplifier drift due to temperature changes. The distortion is proportional to the signal magnitude, meaning it is most visible during dynamic driving conditions and nearly invisible when the true value is close to zero.

\textbf{Noise fault.} Additive Gaussian noise $\eta(t)$ corrupts the sensor output. The noise standard deviation $\sigma$ determines the fault severity. Noise faults are caused by electromagnetic interference (EMI) from nearby components, degraded shielding, or loose electrical connections. Unlike gain faults, noise faults are visible regardless of the signal magnitude, appearing as high-frequency fluctuations superimposed on the true signal.

\textbf{Stuck-at fault.} The sensor output freezes at a constant value $c$ from the fault onset time $t_f$ onward. This represents the most severe fault type, corresponding to a complete sensor failure such as a broken wire, shorted circuit, or failed sensing element. The frozen value may be zero, the last valid reading, or an arbitrary voltage determined by the failure mode.

\subsection{Fault Locations}\label{sec:fault_locations}

In this thesis, faults are injected into two sensor signals:

\begin{enumerate}
    \item \textbf{Accelerator pedal position (\%):} Measures the driver's throttle demand. Range: 0\% (released) to 100\% (fully pressed). A fault in this sensor can cause unintended acceleration or loss of throttle response.
    \item \textbf{Vehicle speed (km/h):} Measures how fast the vehicle is traveling. A fault in this sensor can affect cruise control, ABS, and stability control systems.
\end{enumerate}

The combination of three fault types and two fault locations yields six distinct fault scenarios, each evaluated independently in Chapter~\ref{ch:results}.

\section{Deep Learning for Time-Series Data}\label{sec:dl_timeseries}

This section introduces the deep learning architectures relevant to the encoder network used in this thesis.

\subsection{Convolutional Neural Networks (CNNs)}\label{sec:cnn_background}

A CNN applies learnable convolutional filters to extract local patterns from input data \parencite{lecun2015deep}. For one-dimensional time-series data, a 1D convolutional layer computes:

\begin{equation}\label{eq:conv1d}
    z_j^{(l)}[n] = \sigma\!\left(\sum_{i=1}^{C_{l-1}} \sum_{m=0}^{K-1} w_{ij}^{(l)}[m] \cdot a_i^{(l-1)}[n \cdot s + m] + b_j^{(l)}\right)
\end{equation}

where $a_i^{(l-1)}[n]$ is the $i$-th input channel at position $n$, $w_{ij}^{(l)}[m]$ are the filter weights of length $K$, $s$ is the stride, $b_j^{(l)}$ is the bias term, $\sigma(\cdot)$ is a nonlinear activation function, and $z_j^{(l)}[n]$ is the $j$-th output feature map at position $n$. The number of input channels is $C_{l-1}$ and the number of output channels (filters) is $C_l$.

\subsection{Batch Normalization}\label{sec:batchnorm}

Batch normalization \parencite{ioffe2015batch} normalizes the pre-activation values within each mini-batch to stabilize and accelerate training:

\begin{equation}\label{eq:batchnorm}
    \hat{z}_j = \frac{z_j - \mu_{\mathcal{B}}}{\sqrt{\sigma_{\mathcal{B}}^2 + \epsilon}} \cdot \gamma + \beta
\end{equation}

where $\mu_{\mathcal{B}}$ and $\sigma_{\mathcal{B}}^2$ are the mean and variance computed over the current mini-batch, $\gamma$ and $\beta$ are learnable scale and shift parameters, and $\epsilon$ is a small constant for numerical stability.

\subsection{Pooling Operations}\label{sec:pooling}

Max pooling selects the maximum value within each pooling window of size $p$:

\begin{equation}\label{eq:maxpool}
    a_j^{\text{pool}}[n] = \max_{m=0}^{p-1}\, z_j[n \cdot p + m]
\end{equation}

Global average pooling computes the mean across the entire temporal dimension:

\begin{equation}\label{eq:gap}
    \bar{a}_j = \frac{1}{T} \sum_{n=1}^{T} z_j[n]
\end{equation}

producing a single scalar per feature map. This operation converts variable-length feature maps into a fixed-size representation, which is essential for mapping time-series windows of arbitrary length to fixed-dimensional embeddings.

\section{Self-Supervised Learning}\label{sec:ssl_background}

Self-supervised learning (SSL) is a paradigm in which a model learns useful representations from unlabeled data by solving pretext tasks that are derived from the data's own structure \parencite{jaiswal2021survey}. The supervisory signal comes not from human annotations but from the inherent properties of the data---temporal order, spatial coherence, or augmentation invariance.

SSL methods can be categorized into three families:

\begin{enumerate}
    \item \textbf{Generative methods} learn to reconstruct the input data (e.g., autoencoders, variational autoencoders). The reconstruction objective encourages the model to capture the most informative features of the data.
    \item \textbf{Predictive methods} learn to predict a missing or future part of the input from the available context (e.g., masked language models, Contrastive Predictive Coding \parencite{oord2018representation}).
    \item \textbf{Contrastive methods} learn to distinguish between similar and dissimilar data pairs in a learned embedding space \parencite{chen2020simple, he2020momentum}. This thesis employs contrastive learning.
\end{enumerate}

The advantage of SSL for fault detection is that it requires only healthy data for training. Since healthy sensor recordings are abundantly available from routine test drives and HIL experiments, SSL eliminates the need for expensive labeled fault datasets.

\section{Contrastive Learning}\label{sec:contrastive_learning}

Contrastive learning trains an encoder to produce representations in which similar samples are close together and dissimilar samples are far apart. This section provides the mathematical formulation in detail, as the contrastive objective is the central training mechanism of the proposed framework.

\subsection{Problem Formulation}\label{sec:cl_formulation}

Given an unlabeled dataset $\mathcal{D} = \{x_1, x_2, \ldots, x_N\}$ of $N$ sensor data windows, the goal is to learn an encoder function $f_\theta : \mathcal{X} \rightarrow \mathbb{R}^d$ parameterized by $\theta$ that maps each input $x_i \in \mathcal{X}$ to a $d$-dimensional representation $h_i = f_\theta(x_i)$ such that the representation captures the essential structure of the data.

To achieve this without labels, contrastive learning constructs \emph{positive pairs} (two views of the same sample that should have similar representations) and \emph{negative pairs} (views of different samples that should have dissimilar representations).

\subsection{Data Augmentation}\label{sec:augmentation_theory}

A stochastic augmentation module $\mathcal{T}$ transforms each sample $x_i$ into two correlated views:

\begin{align}
    \tilde{x}_i &= t(x_i), \quad t \sim \mathcal{T} \label{eq:aug1}\\
    \tilde{x}_j &= t'(x_i), \quad t' \sim \mathcal{T} \label{eq:aug2}
\end{align}

where $t$ and $t'$ are independently sampled augmentation functions from the family $\mathcal{T}$. The pair $(\tilde{x}_i, \tilde{x}_j)$ forms a \emph{positive pair}: two different views of the same underlying data that the encoder should map to similar representations.

For time-series data, appropriate augmentations must preserve the semantic content of the signal while introducing sufficient variability. Three augmentation strategies are employed in this thesis (detailed in Section~\ref{sec:augmentation_impl}): Gaussian jittering, amplitude scaling, and temporal masking.

\subsection{Encoder and Projection Head}\label{sec:encoder_projection}

The contrastive framework consists of two components:

\begin{enumerate}
    \item An encoder $f_\theta(\cdot)$ that extracts a representation vector from each augmented view:
    \begin{equation}\label{eq:encoder_def}
        h_i = f_\theta(\tilde{x}_i) \in \mathbb{R}^d
    \end{equation}

    \item A projection head $g_\phi(\cdot)$ that maps the representation to a lower-dimensional space where the contrastive loss is applied:
    \begin{equation}\label{eq:projhead}
        z_i = g_\phi(h_i) = W^{(2)} \cdot \sigma\!\left(W^{(1)} h_i + b^{(1)}\right) + b^{(2)}
    \end{equation}
    where $W^{(1)} \in \mathbb{R}^{d \times d}$, $W^{(2)} \in \mathbb{R}^{p \times d}$ are weight matrices, $b^{(1)}, b^{(2)}$ are bias vectors, $\sigma(\cdot)$ is the ReLU activation, and $p$ is the projection dimension.
\end{enumerate}

A critical finding of \textcite{chen2020simple} is that the contrastive loss should be applied to the projected representations $z_i$ rather than the encoder representations $h_i$, because the projection head can discard information (such as augmentation-specific details) that is useful for the contrastive task but irrelevant for downstream tasks. After training, the projection head is discarded, and only the encoder representations $h_i$ are used.

\subsection{Cosine Similarity}\label{sec:cosine_sim}

The similarity between two projected representations is measured using cosine similarity:

\begin{equation}\label{eq:cossim}
    \mathrm{sim}(z_i, z_j) = \frac{z_i^\top z_j}{\|z_i\|_2 \cdot \|z_j\|_2}
\end{equation}

Cosine similarity measures the angular alignment between two vectors, independent of their magnitudes. It ranges from $-1$ (opposite directions) through $0$ (orthogonal) to $+1$ (identical direction). This metric is preferred over Euclidean distance in contrastive learning because it is invariant to the scale of the representations, which simplifies optimization.

\subsection{NT-Xent Loss}\label{sec:ntxent}

The Normalized Temperature-scaled Cross-Entropy (NT-Xent) loss, introduced by \textcite{chen2020simple}, is the training objective of SimCLR. For a mini-batch of $N$ samples, $2N$ augmented views are generated. For a positive pair $(i, j)$ derived from the same original sample, the loss is:

\begin{equation}\label{eq:ntxent}
    \ell_{i,j} = -\log \frac{\exp\!\big(\mathrm{sim}(z_i, z_j)\,/\,\tau\big)}{\displaystyle\sum_{k=1}^{2N} \mathbbm{1}_{[k \neq i]}\, \exp\!\big(\mathrm{sim}(z_i, z_k)\,/\,\tau\big)}
\end{equation}

where $\tau > 0$ is a temperature parameter and $\mathbbm{1}_{[k \neq i]}$ is an indicator function equal to 1 when $k \neq i$. The numerator contains the similarity of the positive pair, while the denominator sums over all $2N - 1$ other views in the mini-batch, which serve as negative examples.

The total loss over the mini-batch is the average over all positive pairs:

\begin{equation}\label{eq:total_loss}
    \mathcal{L} = \frac{1}{2N} \sum_{k=1}^{N} \Big[\ell_{2k-1,\,2k} + \ell_{2k,\,2k-1}\Big]
\end{equation}

Each original sample contributes two loss terms (one for each view serving as the anchor), ensuring symmetry.

\subsection{Role of the Temperature Parameter}\label{sec:temperature}

The temperature $\tau$ in Equation~\ref{eq:ntxent} controls the sharpness of the similarity distribution in the softmax:

\begin{itemize}
    \item A \textbf{low temperature} (e.g., $\tau = 0.1$) produces a peaked distribution that concentrates the gradient on the hardest negative examples. This encourages fine-grained discrimination but can lead to training instability.
    \item A \textbf{high temperature} (e.g., $\tau = 1.0$) produces a flatter distribution that treats all negatives more equally, yielding smoother gradients but potentially weaker discrimination.
    \item The value $\tau = 0.5$ used in this thesis represents a balance between hard negative mining and stable optimization, following the recommendations of \textcite{chen2020simple}.
\end{itemize}

\subsection{Relationship Between Batch Size and Negative Examples}\label{sec:batchsize}

For a mini-batch of size $N$, each positive pair is contrasted against $2N - 2$ negative pairs. Larger batch sizes therefore provide more diverse negative examples, which has been shown to improve the quality of learned representations \parencite{chen2020simple}. With the batch size $N = 128$ used in this thesis, each sample is compared against 254 negatives per training step.

\subsection{Connection to Noise-Contrastive Estimation}\label{sec:nce}

The NT-Xent loss can be understood as a form of noise-contrastive estimation (NCE) \parencite{gutmann2010noise}. NCE trains a model to distinguish between data samples (positives) and noise samples (negatives) drawn from a known distribution. In the contrastive learning setting, the ``noise'' samples are simply the other augmented views in the mini-batch. The temperature-scaled softmax in Equation~\ref{eq:ntxent} corresponds to a multi-class classification problem where the model must identify the correct positive among all candidates.

\section{Anomaly Detection}\label{sec:anomaly_detection_bg}

After contrastive pretraining, the encoder produces embeddings that capture the manifold of normal sensor behavior. Anomaly detection identifies test samples whose embeddings deviate significantly from this normal manifold.

\subsection{Distance-Based Anomaly Detection}\label{sec:distance_anomaly}

The simplest approach computes the distance (or negative similarity) between a test embedding and a reference point representing the normal distribution. In this thesis, the reference is the centroid of the healthy calibration embeddings:

\begin{equation}\label{eq:centroid}
    \mu_h = \frac{1}{|\mathcal{H}|} \sum_{h_i \in \mathcal{H}} h_i
\end{equation}

where $\mathcal{H}$ is the set of embeddings computed from healthy calibration windows. The anomaly score for a new window with embedding $h_\text{new}$ is:

\begin{equation}\label{eq:anomaly_score}
    s = \mathrm{sim}(h_\text{new},\, \mu_h) = \frac{h_\text{new}^\top \mu_h}{\|h_\text{new}\|_2 \cdot \|\mu_h\|_2}
\end{equation}

A low similarity score indicates that the test window is far from the healthy centroid in the embedding space and is therefore more likely to contain a fault.

\subsection{Threshold-Based Detection}\label{sec:threshold}

The detection decision is made by comparing the anomaly score against a threshold $\theta$:

\begin{equation}\label{eq:decision}
    \text{decision}(h_\text{new}) = \begin{cases}
        \textsc{Faulty} & \text{if } s < \theta \\
        \textsc{Healthy} & \text{if } s \geq \theta
    \end{cases}
\end{equation}

The threshold is set based on a percentile of the healthy similarity distribution. Given the sorted healthy similarities $\{s_1 \leq s_2 \leq \cdots \leq s_{|\mathcal{H}|}\}$, the $p$-th percentile threshold is:

\begin{equation}\label{eq:percentile}
    \theta_p = s_{\lfloor p/100 \cdot |\mathcal{H}| \rfloor}
\end{equation}

Lower percentiles yield more permissive thresholds (only the most extreme deviations are flagged), while higher percentiles yield stricter thresholds (more faults are caught at the cost of more false alarms). This thesis evaluates six percentiles: 15, 20, 25, 30, 35, and~40.

\section{Evaluation Metrics}\label{sec:metrics_bg}

The following metrics are used throughout this thesis for evaluating fault detection performance. All metrics are defined in terms of the confusion matrix entries: true positives (TP), true negatives (TN), false positives (FP), and false negatives (FN).

\begin{table}[H]
\centering
\caption{Confusion matrix for binary fault detection.}
\label{tab:confusion_matrix}
\renewcommand{\arraystretch}{1.3}
\begin{tabular}{l|cc}
\toprule
 & \textbf{Predicted Healthy} & \textbf{Predicted Faulty} \\
\midrule
\textbf{Actually Healthy} & TN & FP \\
\textbf{Actually Faulty}  & FN & TP \\
\bottomrule
\end{tabular}
\end{table}

\textbf{Precision} measures the fraction of detected anomalies that are true faults:
\begin{equation}\label{eq:precision}
    \text{Precision} = \frac{\text{TP}}{\text{TP} + \text{FP}}
\end{equation}

\textbf{Recall} (sensitivity) measures the fraction of actual faults that are correctly detected:
\begin{equation}\label{eq:recall}
    \text{Recall} = \frac{\text{TP}}{\text{TP} + \text{FN}}
\end{equation}

In safety-critical automotive applications, recall is typically prioritized because missing a real fault (FN) poses a greater risk than a false alarm (FP).

\textbf{F1-score} is the harmonic mean of precision and recall:
\begin{equation}\label{eq:f1}
    F_1 = 2 \cdot \frac{\text{Precision} \cdot \text{Recall}}{\text{Precision} + \text{Recall}}
\end{equation}

\textbf{Accuracy} is the fraction of all predictions that are correct:
\begin{equation}\label{eq:accuracy}
    \text{Accuracy} = \frac{\text{TP} + \text{TN}}{\text{TP} + \text{TN} + \text{FP} + \text{FN}}
\end{equation}

\textbf{ROC curve and AUC.} The Receiver Operating Characteristic (ROC) curve plots the true positive rate (recall) against the false positive rate ($\text{FPR} = \text{FP}/(\text{FP}+\text{TN})$) across all possible threshold values. The Area Under the ROC Curve (AUC) provides a single scalar measure of discrimination ability: AUC~$=0.5$ indicates random performance, AUC~$=1.0$ indicates perfect separation.

\section{Summary}\label{sec:bg_summary}

This chapter has established the theoretical foundations for the proposed framework. The automotive V-model and HIL testing provide the application context and the source of both training and test data. The mathematical formulations of contrastive learning---including data augmentation, the NT-Xent loss, and the role of the temperature parameter---define the self-supervised training objective. Cosine similarity-based anomaly detection with percentile thresholding provides the fault detection mechanism. The evaluation metrics (precision, recall, F1, accuracy, confusion matrix, ROC/AUC) enable a rigorous assessment of detection performance. These components are combined in the methodology presented in Chapter~\ref{ch:methodology}.

    \chapter{Related Work}\label{ch:related_work}

This chapter reviews current research at the intersection of self-supervised learning, contrastive representation learning, and fault detection in industrial and automotive systems. The review is organized thematically, beginning with SSL methods for fault diagnosis, followed by contrastive learning frameworks, anomaly detection with learned representations, and deep learning for automotive applications. Each section identifies the contributions and limitations of existing work. The chapter concludes with a summary of research gaps that motivate the approach presented in this thesis.

\section{Self-Supervised Learning for Fault Diagnosis}\label{sec:rw_ssl_fd}

The application of self-supervised learning to fault diagnosis has gained significant attention in recent years, driven by the practical difficulty of obtaining labeled fault data in industrial settings.

\textcite{xu2024self} present a comprehensive review of SSL methods for machinery fault diagnosis, covering contrastive, generative, and predictive approaches. Their survey of over 100 publications identifies contrastive learning as the most effective paradigm for downstream fault classification, particularly when labeled data is scarce. The authors note that most SSL-based methods have been evaluated on benchmark bearing datasets (CWRU, Paderborn) rather than on automotive sensor data, and that the cross-domain transfer capability of these methods remains underexplored.

\textcite{wang2023self} propose a self-supervised framework for bearing fault diagnosis that combines contrastive learning with wavelet-based time-frequency analysis. The method achieves 95.3\% accuracy on the CWRU bearing dataset when fine-tuned with only 10\% of the available labels. While demonstrating the value of SSL pretraining for reducing label requirements, the approach still relies on a labeled fine-tuning stage to adapt the learned representations to specific fault classes. In contrast, the framework proposed in this thesis operates without any labeled fault data, using the SSL-learned representations directly for anomaly detection.

\textcite{ding2022self} introduce a self-supervised pretraining method based on contrastive predictive coding (CPC) for incipient fault detection in bearings. CPC learns by predicting future signal segments in the latent space, capturing temporal dependencies that are informative for early-stage fault detection. The approach achieves strong results on vibration data, detecting subtle faults before they become severe. However, CPC is designed for sequential prediction and may not capture the multi-scale patterns present in automotive sensor signals, where different fault types manifest at different temporal scales.

\textcite{li2023contrastive} address the specific challenge of fault diagnosis under limited labeled data by combining contrastive learning with a semi-supervised classifier. Their method, evaluated on rolling bearing data, uses contrastive pretraining to learn general signal features, followed by a classifier trained on a small labeled subset. The authors report accuracy above 90\% with as few as 10 labeled samples per fault class. While the semi-supervised strategy is practical, the requirement for even a small labeled set poses challenges in scenarios where certain fault types have never been observed.

\textcite{zhang2023self_vibration} apply self-supervised learning to vibration signal analysis, proposing a masked autoencoder that reconstructs randomly masked portions of the input signal. The reconstructive pretext task forces the encoder to learn the underlying signal dynamics, and the resulting representations transfer well to downstream fault classification. The method achieves competitive performance under label-scarce conditions, but the generative reconstruction objective may not produce embeddings that are as discriminative for anomaly detection as those learned through contrastive objectives.

\section{Contrastive Learning Frameworks}\label{sec:rw_cl}

The development of contrastive learning methods has progressed rapidly, with several landmark frameworks shaping the field.

\textcite{chen2020simple} introduced SimCLR, demonstrating that strong contrastive representations can be learned with a simple framework consisting of data augmentations, a base encoder, a projection head, and the NT-Xent loss. Key findings include: (1)~composition of multiple data augmentations is critical, (2)~a learnable nonlinear projection head between the representation and the loss substantially improves quality, (3)~larger batch sizes benefit contrastive learning by providing more negative examples, and (4)~the temperature parameter in the NT-Xent loss must be carefully tuned. SimCLR achieves 76.5\% top-1 accuracy on ImageNet under linear evaluation, matching a supervised ResNet-50. The simplicity and effectiveness of SimCLR make it a natural starting point for adapting contrastive learning to new domains, including time-series data.

\textcite{he2020momentum} propose Momentum Contrast (MoCo), which addresses the batch size limitation of SimCLR by maintaining a large, dynamically updated dictionary of negative examples through a momentum-updated encoder. MoCo decouples the dictionary size from the mini-batch size, enabling contrastive learning with standard batch sizes on commodity hardware. While MoCo offers practical advantages for resource-constrained settings, SimCLR's simpler architecture is preferred in this thesis for clarity and reproducibility.

\textcite{grill2020bootstrap} introduce Bootstrap Your Own Latent (BYOL), which challenges the assumption that negative examples are necessary for contrastive learning. BYOL uses an asymmetric architecture with a student network and a momentum-updated teacher network, learning by making the student's representations match the teacher's without any negative pairs. This eliminates the batch size sensitivity inherent in SimCLR but introduces additional architectural complexity. The absence of negative pairs also makes it less straightforward to interpret the learned embedding space for anomaly detection purposes.

\textcite{oord2018representation} propose Contrastive Predictive Coding (CPC), a framework that learns representations by predicting future observations in the latent space. CPC is particularly relevant for sequential data because it explicitly models temporal structure through an autoregressive architecture. However, the directional (past-to-future) prediction may not capture all patterns relevant to fault detection, where anomalies can affect any part of a window.

For time-series applications specifically, \textcite{yue2022ts2vec} propose TS2Vec, which performs contrastive learning hierarchically over augmented context views at multiple temporal scales. TS2Vec achieves state-of-the-art results on time-series classification, forecasting, and anomaly detection benchmarks by capturing both local and global temporal patterns. \textcite{eldele2021time} introduce TS-TCC, which designs two augmentation strategies for temporal data---weak augmentation (jittering, scaling) and strong augmentation (permutation, masking)---and applies contrastive learning between the resulting views. These time-series-specific frameworks inform the augmentation design used in this thesis.

\textcite{zhang2022self} propose a contrastive learning method that enforces consistency between time-domain and frequency-domain representations of the same signal. By contrasting signals in both domains, the encoder learns features that capture both temporal dynamics and spectral characteristics. While this dual-domain approach is promising, the automotive sensor signals in this thesis (accelerator position, vehicle speed) are relatively low-frequency, and the additional complexity of frequency-domain processing is not justified.

\section{Anomaly Detection with Self-Supervised Representations}\label{sec:rw_anomaly}

The combination of self-supervised representation learning with anomaly detection methods has emerged as a powerful paradigm for identifying out-of-distribution samples.

\textcite{ruff2021unifying} provide a unifying theoretical framework for deep anomaly detection, establishing connections between one-class classification, contrastive learning, and classical shallow methods (one-class SVM, isolation forests). Their analysis demonstrates that self-supervised pretraining followed by simple anomaly scoring in the learned feature space consistently outperforms end-to-end trained anomaly detectors. This finding supports the two-stage approach used in this thesis: contrastive pretraining followed by cosine similarity-based scoring.

\textcite{pang2021deep} review deep learning methods for anomaly detection and identify three fundamental challenges: (1)~learning effective anomaly representations with limited or no labeled anomalies, (2)~detecting complex anomalies that affect multiple features simultaneously, and (3)~providing interpretable detection results. The first two challenges are directly addressed by the framework in this thesis---contrastive pretraining eliminates the need for labeled anomalies, and the multi-sensor windows capture cross-sensor fault effects.

\textcite{liu2023self} survey self-supervised methods specifically designed for anomaly detection, distinguishing between methods that model the data distribution directly and those that learn transformation-invariant representations. They identify contrastive learning as particularly suitable for anomaly detection because the learned embeddings naturally separate in-distribution (normal) and out-of-distribution (anomalous) samples.

\section{Deep Learning for Automotive Fault Detection}\label{sec:rw_automotive}

Fault detection in automotive systems has traditionally relied on model-based methods (observer-based, parity equation, Kalman filter) and threshold monitoring. The adoption of deep learning has expanded the scope and accuracy of data-driven detection.

\textcite{zhao2024deep} review deep learning approaches for automotive sensor fault detection, covering CNNs, RNNs, LSTM networks, and hybrid architectures. The authors identify the scarcity of labeled fault data as the primary barrier to wider adoption and suggest self-supervised and transfer learning as promising directions for future research. This thesis directly addresses this identified gap.

\textcite{abboush2022intelligent} propose a hybrid deep learning framework combining CNNs for spatial feature extraction with GRUs for temporal modeling, applied to HIL test data from a dSPACE simulator with a gasoline engine model. The method achieves high classification accuracy for both fault type and fault location identification. However, the approach is fully supervised, requiring labeled examples of every fault type and location for training. The HIL testing infrastructure and fault injection methodology developed in that work provide the foundation for the test data used in this thesis.

\textcite{ghannoum2025explainable} extend the work of \textcite{abboush2022intelligent} by integrating Explainable AI (XAI) techniques---Integrated Gradients, DeepLIFT, Gradient SHAP, and DeepLIFT SHAP---into the fault detection pipeline. The Fault Location Model (FLM) achieves 97.40\% accuracy, and the Fault Type Model (FTM) achieves 97.19\% accuracy. The XAI analysis identifies the most important features for each fault class and reduces the feature set from 24 to 10 with minimal accuracy loss. While achieving impressive results, this work requires fully labeled datasets and cannot detect fault types not present in the training data.

\textcite{chen2023transfer} survey transfer learning for bearing fault diagnosis, finding that domain adaptation methods can bridge the gap between source and target data distributions. Their review highlights the practical scenario where models trained on laboratory data must be deployed on real-world equipment---analogous to the A2D2-to-HIL transfer scenario addressed in this thesis.

\textcite{tang2024self} review self-supervised learning for automotive perception tasks, covering camera-based and LiDAR-based applications. While focused primarily on perception rather than fault detection, this work demonstrates the viability of SSL methods in the automotive domain and identifies data augmentation design as a critical factor for success.

\section{Review Summary and Research Gaps}\label{sec:review_gaps}

Table~\ref{tab:related_work_comparison} provides a structured comparison of the reviewed approaches against the proposed method.

\begin{table}[H]
\centering
\caption{Comparison of the proposed approach with existing methods.}
\label{tab:related_work_comparison}
\renewcommand{\arraystretch}{1.3}
\small
\begin{tabular}{p{3.2cm} c c c c c}
\toprule
\textbf{Method} & \textbf{SSL} & \textbf{No Labels} & \textbf{Automotive} & \textbf{HIL} & \textbf{Transfer} \\
\midrule
Abboush et al. \parencite{abboush2022intelligent} & \texttimes & \texttimes & \checkmark & \checkmark & \texttimes \\
Ghannoum \parencite{ghannoum2025explainable} & \texttimes & \texttimes & \checkmark & \checkmark & \texttimes \\
Wang et al. \parencite{wang2023self} & \checkmark & \texttimes\textsuperscript{*} & \texttimes & \texttimes & \texttimes \\
Ding et al. \parencite{ding2022self} & \checkmark & \texttimes\textsuperscript{*} & \texttimes & \texttimes & \texttimes \\
Li et al. \parencite{li2023contrastive} & \checkmark & \texttimes\textsuperscript{*} & \texttimes & \texttimes & \texttimes \\
Zhang et al. \parencite{zhang2023self_vibration} & \checkmark & \texttimes\textsuperscript{*} & \texttimes & \texttimes & \texttimes \\
\textbf{This thesis} & \checkmark & \checkmark & \checkmark & \checkmark & \checkmark \\
\bottomrule
\end{tabular}
\\[0.5em]
\textsuperscript{*}Requires labeled data for fine-tuning or classifier training.
\end{table}

The following research gaps are identified from the literature review:

\begin{enumerate}
    \item \textbf{Domain limitation.} The majority of SSL-based fault detection research targets rotating machinery (bearings, motors) using vibration signals. Applications to automotive sensor systems remain scarce.

    \item \textbf{Residual label dependency.} Most SSL approaches reduce rather than eliminate the need for labeled fault data, requiring at least a small labeled set for fine-tuning or semi-supervised classification.

    \item \textbf{Cross-domain validation.} Few works evaluate the transfer of SSL-learned representations between different data acquisition systems (e.g., from real test drives to HIL simulators), which is a critical practical requirement for deployment.

    \item \textbf{Threshold analysis.} Detection performance is typically reported at a single operating point, without systematically exploring the effect of threshold selection on the precision--recall trade-off.

    \item \textbf{Incomplete evaluation.} Many studies report only accuracy, omitting precision, recall, F1-score, confusion matrices, and ROC analysis that are essential for assessing performance in safety-critical applications.

    \item \textbf{Missing cost analysis.} Computational cost analysis (training time, inference time) is frequently absent, making it difficult to assess the practical feasibility of deployment.
\end{enumerate}

This thesis addresses all six gaps: it applies contrastive SSL to automotive sensors, requires zero labeled fault data, demonstrates A2D2-to-HIL transfer, evaluates six threshold settings, reports comprehensive metrics for all experiments, and measures computational costs for all pipeline stages.

    \chapter{Methodology, Implementation, and Results}\label{ch:methodology}

This chapter presents the complete methodology, implementation details, and intermediate results of the proposed self-supervised fault detection framework. Each subsection is described with sufficient detail for reproducibility by other researchers. The chapter begins with the overall system architecture and then proceeds through each component in the order of the processing pipeline.

\section{System Architecture}\label{sec:architecture}

Figure~\ref{fig:architecture} illustrates the three-phase architecture of the proposed framework.

\begin{figure}[H]
\centering
\fbox{\parbox{0.92\textwidth}{\centering\vspace{5cm}
\textbf{[PLACEHOLDER -- Figure~\ref{fig:architecture}]}\\[0.5em]
Generate a high-level architecture diagram showing the three phases:\\
\textbf{Phase~1:} A2D2 Healthy Data $\rightarrow$ Preprocessing $\rightarrow$ Normalization $\rightarrow$ Windowing $\rightarrow$ Augmentation $\rightarrow$ SimCLR 1D-CNN Encoder Training\\
\textbf{Phase~2:} HIL Healthy Data (90\,s) $\rightarrow$ Frozen Encoder $\rightarrow$ Healthy Centroid + Threshold Calibration\\
\textbf{Phase~3:} HIL Fault Data $\rightarrow$ Frozen Encoder $\rightarrow$ Cosine Similarity $\rightarrow$ Threshold Comparison $\rightarrow$ Healthy/Faulty Decision\\
Use a flowchart style similar to Figure~9 in the Ehab thesis.
\vspace{0.5cm}}}
\caption{Three-phase architecture of the proposed SSL-based fault detection framework.}
\label{fig:architecture}
\end{figure}

The framework consists of three phases, each with clearly defined inputs, processing steps, and outputs:

\begin{enumerate}
    \item \textbf{Phase~1: Self-Supervised Pretraining} (Sections~\ref{sec:a2d2_loading}--\ref{sec:simclr_impl}). The 1D-CNN encoder is trained using the SimCLR contrastive learning objective on healthy sensor data from the A2D2 dataset. This phase runs once and produces a trained encoder that captures the normal behavior of automotive sensors.

    \item \textbf{Phase~2: Anomaly Detection Calibration} (Section~\ref{sec:calibration_impl}). A small amount of healthy HIL data is passed through the frozen encoder to establish a reference distribution. The healthy centroid and similarity-based detection thresholds are computed.

    \item \textbf{Phase~3: Fault Detection and Evaluation} (Section~\ref{sec:detection_impl}). HIL data containing injected faults is evaluated against the calibrated thresholds. Comprehensive metrics are computed for each fault type and threshold setting.
\end{enumerate}

\section{Data Collection and Sources}\label{sec:data_sources}

Two data sources are used in this thesis, reflecting the cross-domain transfer scenario from real-world driving to HIL simulation.

\subsection{A2D2 Dataset (Training Data)}\label{sec:a2d2_loading}

The Audi Autonomous Driving Dataset \parencite{a2d2_2020} provides the training data. The raw data is stored in JSON files named \texttt{bus\_signals.json}, each containing timestamped sensor recordings from a real test drive. The loading procedure is as follows:

\begin{enumerate}
    \item \textbf{File discovery:} Search the data directory recursively for \texttt{bus\_signals.json} files larger than 1~MB (to exclude incomplete header-only files).
    \item \textbf{JSON parsing:} For each file, extract the \texttt{accelerator\_pedal} and \texttt{vehicle\_speed} entries, each containing a list of $[\text{timestamp}_{\mu s},\, \text{value}]$ pairs.
    \item \textbf{Sampling rate verification:} Compute the actual sampling rate from consecutive timestamp differences. The accelerator pedal is recorded at approximately 100~Hz and the vehicle speed at approximately 50~Hz.
    \item \textbf{Speed upsampling:} The vehicle speed signal is upsampled from 50~Hz to 100~Hz using linear interpolation (\texttt{scipy.interpolate.interp1d}) to align with the accelerator timestamps. Linear interpolation is appropriate because vehicle speed changes slowly due to the vehicle's inertia, and the interpolation interval (0.02~s) is short relative to the speed dynamics.
    \item \textbf{Dataset concatenation:} Data from up to three JSON files is concatenated into a single DataFrame containing approximately 219,064~samples (36.5~minutes at 100~Hz).
\end{enumerate}

Table~\ref{tab:a2d2_stats} reports the statistical properties of the combined A2D2 data before normalization.

\begin{table}[H]
\centering
\caption{Statistical properties of the A2D2 training data before normalization.}
\label{tab:a2d2_stats}
\renewcommand{\arraystretch}{1.3}
\begin{tabular}{lcccccc}
\toprule
\textbf{Sensor} & \textbf{Unit} & \textbf{Samples} & \textbf{Rate (Hz)} & \textbf{Mean} & \textbf{Std} & \textbf{Range} \\
\midrule
Accelerator & \% & 219{,}064 & 100 & 6.99 & 8.70 & 0.00--52.00 \\
Speed & km/h & 219{,}064 & 50$\rightarrow$100 & 17.21 & 14.78 & 0.00--73.36 \\
\bottomrule
\end{tabular}\\[0.3em]
\small The Pearson correlation between the two sensors is $r = 0.527$.
\end{table}

\begin{figure}[H]
\centering
\includegraphics[width=0.95\textwidth]{figures/part1_a2d2_comprehensive.png}
\caption{A2D2 data analysis: time-series signals, distribution histograms, box plots, and sensor correlation for the combined training dataset.}
\label{fig:a2d2_signals}
\end{figure}

\subsection{HIL Dataset (Calibration and Test Data)}\label{sec:hil_data}

The HIL data originates from a dSPACE Hardware-in-the-Loop test bench at TU~Clausthal \parencite{abboush2022intelligent}. It comprises one healthy recording and six fault-injected recordings, each stored as a CSV file exported from the HIL data logger.

\textbf{HIL data parsing.} The CSV files follow a specific format originating from the dSPACE measurement data format (MDF). The parsing procedure identifies the header row (beginning with ``path,''), locates the sensor columns by keyword matching (``AccPedal'' or ``accelerator'' for the pedal position; ``v\_Vehicle'' or ``speed'' for the vehicle speed), and extracts the numerical data rows (beginning with ``trace\_values,''). Missing values are handled by linear interpolation.

Table~\ref{tab:hil_files} lists all HIL data files used in this thesis.

\begin{table}[H]
\centering
\caption{Overview of HIL data files used for calibration and testing.}
\label{tab:hil_files}
\renewcommand{\arraystretch}{1.3}
\begin{tabular}{llllc}
\toprule
\textbf{File} & \textbf{Usage} & \textbf{Faulty Sensor} & \textbf{Fault Type} & \textbf{$\sim$Samples} \\
\midrule
healthy.csv & Calibration & -- & -- & 31{,}881 (first 9{,}000 used) \\
acc fault gain.csv & Test & Accelerator & Gain & 31{,}546 \\
acc fault noise.csv & Test & Accelerator & Noise & 32{,}205 \\
acc fault stuck.csv & Test & Accelerator & Stuck-at & 31{,}684 \\
rpm fault gain.csv & Test & Speed & Gain & 32{,}276 \\
rpm fault noise.csv & Test & Speed & Noise & 31{,}647 \\
rpm fault stuck at.csv & Test & Speed & Stuck-at & 32{,}653 \\
\bottomrule
\end{tabular}
\end{table}

\subsection{Dataset Fusion: A2D2 with HIL}\label{sec:dataset_fusion}

A central aspect of this work is the fusion of two data sources with different origins, formats, and characteristics. Table~\ref{tab:fusion_comparison} summarizes the differences.

\begin{table}[H]
\centering
\caption{Comparison of the A2D2 and HIL datasets.}
\label{tab:fusion_comparison}
\renewcommand{\arraystretch}{1.3}
\begin{tabular}{lll}
\toprule
\textbf{Property} & \textbf{A2D2} & \textbf{HIL} \\
\midrule
Source & Real vehicle (Audi, Germany) & dSPACE HIL simulator \\
Format & JSON with $\mu$s timestamps & CSV (MDF export) \\
Content & Healthy driving only & Healthy + 6 fault scenarios \\
Common sensors & accelerator\_pedal, vehicle\_speed & AccPedal, v\_Vehicle \\
Acc.\ sampling rate & $\sim$100~Hz & 100~Hz \\
Speed sampling rate & $\sim$50~Hz (upsampled to 100~Hz) & 100~Hz \\
Purpose & SimCLR training (Phase~1) & Calibration \& testing (Phases~2--3) \\
\bottomrule
\end{tabular}
\end{table}

The fusion process ensures that the encoder, trained on A2D2 data, can be applied to HIL data without retraining:

\begin{enumerate}
    \item \textbf{Sensor alignment:} The common sensors (accelerator, speed) are identified in both datasets and extracted using the respective parsing procedures.
    \item \textbf{Sampling rate alignment:} Both datasets are brought to a uniform 100~Hz rate (A2D2 speed is upsampled; HIL is already at 100~Hz).
    \item \textbf{Normalization alignment:} The Z-score normalization scaler is fitted on the A2D2 training data and then applied to both A2D2 and HIL data (see Section~\ref{sec:normalization_impl}).
    \item \textbf{Windowing alignment:} Identical window size (200~samples) and stride (100~samples) are used for both datasets.
\end{enumerate}

\begin{figure}[H]
\centering
\includegraphics[width=0.95\textwidth]{figures/dataset_fusion_comparison.png}
\caption{Dataset fusion visualization: A2D2 healthy speed signal (top) and HIL speed signal with gain fault region highlighted in red (bottom).}
\label{fig:fusion_plot}
\end{figure}

\section{Data Preprocessing}\label{sec:preprocessing_impl}

\subsection{Z-Score Normalization}\label{sec:normalization_impl}

The combined A2D2 training data is normalized using Z-score standardization:

\begin{equation}\label{eq:zscore_impl}
    x_{\text{norm}}^{(j)} = \frac{x^{(j)} - \mu_j}{\sigma_j}
\end{equation}

where $\mu_j$ and $\sigma_j$ are the mean and standard deviation of the $j$-th sensor, computed exclusively from the A2D2 training data. The scaler parameters are saved and reused to normalize the HIL data, ensuring that both domains are mapped to a comparable feature space. This is critical for cross-domain transfer: without consistent normalization, the encoder would produce incompatible embeddings for A2D2 and HIL inputs.

\subsection{Sliding Window Segmentation}\label{sec:windowing_impl}

The normalized time series is divided into overlapping fixed-length windows using a sliding window approach:

\begin{equation}\label{eq:sliding_window}
    W_k = \big[x[k \cdot S],\; x[k \cdot S + 1],\; \ldots,\; x[k \cdot S + L - 1]\big], \quad k = 0, 1, 2, \ldots
\end{equation}

where $L = 200$ is the window length and $S = 100$ is the stride. At 100~Hz, each window covers 2~seconds of driving, which is sufficient to capture meaningful driving dynamics (acceleration, braking, gear changes). The 50\% overlap ($S = L/2$) generates approximately twice as many training windows compared to non-overlapping segmentation, increasing the diversity of training examples for SimCLR.

Each window $W_k \in \mathbb{R}^{200 \times 2}$ is a matrix with 200~rows (time steps) and 2~columns (accelerator, speed). For input to the 1D-CNN encoder, the dimensions are transposed to $(2, 200)$, placing the sensor channels in the first dimension and the temporal dimension second, following the PyTorch convention for 1D convolutions.

Algorithm~\ref{alg:windowing} formalizes the windowing procedure.

\begin{algorithm}[H]
\caption{Sliding Window Segmentation}
\label{alg:windowing}
\begin{algorithmic}[1]
\State \textbf{Input:} Normalized time series $X \in \mathbb{R}^{T \times C}$, window length $L$, stride $S$
\State \textbf{Output:} Set of windows $\{W_0, W_1, \ldots, W_{K-1}\}$
\State $K \leftarrow \lfloor (T - L) / S \rfloor + 1$
\For{$k = 0$ to $K-1$}
    \State $\text{start} \leftarrow k \cdot S$
    \State $W_k \leftarrow X[\text{start} : \text{start} + L,\; :]$
\EndFor
\State \Return $\{W_0, W_1, \ldots, W_{K-1}\}$
\end{algorithmic}
\end{algorithm}

\section{SimCLR Training Implementation}\label{sec:simclr_impl}

\subsection{Data Augmentation Strategies}\label{sec:augmentation_impl}

Three augmentation strategies are applied independently to generate two views of each window during SimCLR training. Each augmentation is applied with a probability of 50\% per view, ensuring diverse augmentation combinations.

\textbf{Gaussian jittering.} Adds random noise to simulate minor sensor fluctuations:
\begin{equation}\label{eq:jitter_impl}
    \tilde{x}(t) = x(t) + \epsilon(t), \quad \epsilon(t) \sim \mathcal{N}(0, \sigma_j^2)
\end{equation}
with $\sigma_j = 0.1$. This augmentation teaches the encoder to be invariant to small noise-like perturbations that may arise from normal sensor operation.

\textbf{Amplitude scaling.} Multiplies the signal by a random factor:
\begin{equation}\label{eq:scaling_impl}
    \tilde{x}(t) = s \cdot x(t), \quad s \sim \text{Uniform}(0.8,\; 1.2)
\end{equation}
This augmentation produces invariance to minor amplitude variations, which is important because different sensor instances may have slightly different gains even under normal conditions.

\textbf{Temporal masking.} Sets a contiguous segment to zero:
\begin{equation}\label{eq:masking_impl}
    \tilde{x}(t) = \begin{cases} 0 & \text{if } t_0 \leq t < t_0 + m \\ x(t) & \text{otherwise} \end{cases}
\end{equation}
where $t_0 \sim \text{Uniform}(0, L - m)$ is a random start position and $m = \lfloor 0.1 \cdot L \rfloor = 20$ samples (10\% of the window). Masking encourages the encoder to form representations based on the overall signal context rather than relying on any single temporal segment.

\begin{figure}[H]
\centering
\includegraphics[width=0.95\textwidth]{figures/part2_augmentation_examples.png}
\caption{Examples of the three data augmentation strategies applied to a training window. Top: accelerator signal. Bottom: speed signal.}
\label{fig:augmentation_examples}
\end{figure}

\subsection{1D-CNN Encoder Architecture}\label{sec:encoder_impl}

The encoder is a three-block 1D Convolutional Neural Network. Each block consists of a Conv1d layer, batch normalization, ReLU activation, and max pooling. Table~\ref{tab:encoder_layers} provides the layer-by-layer architecture summary.

\begin{table}[H]
\centering
\caption{Layer-wise summary of the 1D-CNN encoder architecture.}
\label{tab:encoder_layers}
\renewcommand{\arraystretch}{1.3}
\begin{tabular}{llccccc}
\toprule
\textbf{Block} & \textbf{Layer} & \textbf{In Ch} & \textbf{Out Ch} & \textbf{Kernel} & \textbf{Stride} & \textbf{Output} \\
\midrule
-- & Input & -- & -- & -- & -- & $(B, 2, 200)$ \\
\midrule
\multirow{3}{*}{Block~1} & Conv1d & 2 & 64 & 7 & 2 & $(B, 64, 100)$ \\
 & BatchNorm + ReLU & -- & -- & -- & -- & $(B, 64, 100)$ \\
 & MaxPool1d & -- & -- & 2 & 2 & $(B, 64, 50)$ \\
\midrule
\multirow{3}{*}{Block~2} & Conv1d & 64 & 128 & 5 & 2 & $(B, 128, 25)$ \\
 & BatchNorm + ReLU & -- & -- & -- & -- & $(B, 128, 25)$ \\
 & MaxPool1d & -- & -- & 2 & 2 & $(B, 128, 12)$ \\
\midrule
\multirow{3}{*}{Block~3} & Conv1d & 128 & 256 & 3 & 1 & $(B, 256, 12)$ \\
 & BatchNorm + ReLU & -- & -- & -- & -- & $(B, 256, 12)$ \\
 & MaxPool1d & -- & -- & 2 & 2 & $(B, 256, 6)$ \\
\midrule
-- & AdaptiveAvgPool1d & -- & -- & -- & -- & $(B, 256, 1)$ \\
-- & Squeeze & -- & -- & -- & -- & $(B, 256)$ \\
\bottomrule
\end{tabular}\\[0.3em]
$B$: batch size.
\end{table}

The encoder takes input tensors of shape $(B, 2, 200)$---batch size $\times$ sensor channels $\times$ time steps---and produces 256-dimensional embedding vectors. The three convolutional blocks progressively increase the feature dimension ($2 \rightarrow 64 \rightarrow 128 \rightarrow 256$) while reducing the temporal dimension through strided convolutions and max pooling. Adaptive global average pooling at the end collapses the remaining temporal dimension into a single value per feature map, producing a fixed-size output regardless of the precise input length.

\subsection{Projection Head}\label{sec:projhead_impl}

The projection head is a two-layer fully connected network:

\begin{equation}
    g(h) = W^{(2)} \cdot \text{ReLU}\!\big(\text{BN}(W^{(1)} \cdot h + b^{(1)})\big) + b^{(2)}
\end{equation}

with $W^{(1)} \in \mathbb{R}^{256 \times 256}$ and $W^{(2)} \in \mathbb{R}^{128 \times 256}$, producing 128-dimensional projections. Following \textcite{chen2020simple}, the projection head is used only during SimCLR training and discarded for downstream anomaly detection.

\subsection{Training Configuration}\label{sec:training_config}

Table~\ref{tab:hyperparameters} lists all hyperparameters used during training.

\begin{table}[H]
\centering
\caption{Hyperparameters for SimCLR training.}
\label{tab:hyperparameters}
\renewcommand{\arraystretch}{1.3}
\begin{tabular}{lll}
\toprule
\textbf{Hyperparameter} & \textbf{Value} & \textbf{Justification} \\
\midrule
Window size $L$ & 200 & 2\,s at 100\,Hz; captures driving dynamics \\
Stride $S$ & 100 & 50\% overlap for data augmentation \\
Batch size $N$ & 128 & 254 negative pairs per sample \\
Epochs & 50 & Experimentally determined convergence \\
Learning rate & $10^{-3}$ & Standard for Adam optimizer \\
Weight decay & $10^{-4}$ & L2 regularization \\
Temperature $\tau$ & 0.5 & Balance of discrimination and stability \\
Embedding dim $d$ & 256 & Encoder output dimension \\
Projection dim $p$ & 128 & NT-Xent loss space dimension \\
Jitter $\sigma_j$ & 0.1 & Minor noise invariance \\
Scale range & $[0.8,\; 1.2]$ & Minor gain invariance \\
Mask ratio $r$ & 0.1 & 10\% temporal masking \\
Optimizer & Adam \parencite{kingma2014adam} & Adaptive per-parameter learning rate \\
\bottomrule
\end{tabular}
\end{table}

\subsection{Training Procedure}\label{sec:training_procedure}

Algorithm~\ref{alg:simclr} formalizes the SimCLR training procedure.

\begin{algorithm}[H]
\caption{SimCLR Training for Automotive Sensor Data}
\label{alg:simclr}
\begin{algorithmic}[1]
\State \textbf{Input:} Training windows $\{W_k\}_{k=1}^{K}$, epochs $E$, batch size $N$, temperature $\tau$
\State \textbf{Output:} Trained encoder $f_\theta$
\State Initialize encoder $f_\theta$ and projection head $g_\phi$
\State Initialize Adam optimizer with lr $= 10^{-3}$, weight decay $= 10^{-4}$
\For{$\text{epoch} = 1$ to $E$}
    \For{each mini-batch $\mathcal{B} = \{W_1, \ldots, W_N\}$}
        \For{each $W_k \in \mathcal{B}$}
            \State $\tilde{W}_k^{(1)} \leftarrow \text{Augment}(W_k)$ \Comment{View 1}
            \State $\tilde{W}_k^{(2)} \leftarrow \text{Augment}(W_k)$ \Comment{View 2}
        \EndFor
        \State $h_k^{(v)} \leftarrow f_\theta(\tilde{W}_k^{(v)})$ for all $k, v$ \Comment{Encode}
        \State $z_k^{(v)} \leftarrow g_\phi(h_k^{(v)})$ for all $k, v$ \Comment{Project}
        \State Compute $\mathcal{L}$ using Eq.~\ref{eq:total_loss} \Comment{NT-Xent loss}
        \State Update $\theta, \phi$ via backpropagation
    \EndFor
    \State Record epoch loss
\EndFor
\State Discard $g_\phi$; save $f_\theta$
\State \Return $f_\theta$
\end{algorithmic}
\end{algorithm}

\begin{figure}[H]
\centering
\includegraphics[width=0.95\textwidth]{figures/part2_training_curves.png}
\caption{SimCLR training loss over 50 epochs. Left: loss at each training step. Right: average loss per epoch.}
\label{fig:training_loss}
\end{figure}

\section{Anomaly Detection Calibration}\label{sec:calibration_impl}

After SimCLR training, the encoder $f_\theta$ is frozen (no further weight updates). The calibration phase establishes the healthy reference distribution in the embedding space.

\subsection{Healthy Calibration Data}\label{sec:healthy_cal}

Only the first 90~seconds of the healthy HIL recording (\texttt{healthy.csv}) is used for calibration. At 100~Hz, this yields approximately 9,000~samples, which are segmented into approximately 88~windows (with stride~100). This small calibration set is sufficient to estimate the healthy centroid and similarity distribution, demonstrating the practical feasibility of the approach: only a brief healthy recording from the target system is needed.

\subsection{Embedding Extraction and Centroid Computation}\label{sec:centroid_impl}

Each calibration window is passed through the frozen encoder to obtain a 256-dimensional embedding. The healthy centroid is computed as the mean of all calibration embeddings:

\begin{equation}
    \mu_h = \frac{1}{|\mathcal{H}|} \sum_{h_i \in \mathcal{H}} h_i
\end{equation}

The cosine similarity between each healthy embedding and the centroid is computed, yielding the healthy similarity distribution. The detection thresholds are set at the 15th, 20th, 25th, 30th, 35th, and 40th percentiles of this distribution.

\section{Fault Detection and Evaluation}\label{sec:detection_impl}

\subsection{Per-Fault Evaluation}\label{sec:per_fault_eval}

Each of the six fault files is processed independently:

\begin{enumerate}
    \item The fault data is loaded, normalized using the saved A2D2 scaler, and segmented into windows.
    \item Each window is passed through the frozen encoder to obtain its embedding.
    \item The cosine similarity between the embedding and the healthy centroid is computed.
    \item The similarity is compared against each of the six thresholds.
    \item For each threshold, the window is classified as ``healthy'' (above threshold) or ``faulty'' (below threshold).
    \item Precision, recall, F1-score, accuracy, and confusion matrix entries are computed.
\end{enumerate}

Since the entire fault file is known to contain faulty data, all windows are labeled as ``faulty'' for ground truth purposes. The metrics therefore measure how well the system detects the presence of the injected fault within the data.

\subsection{Binary Classification Evaluation}\label{sec:binary_eval}

In addition to the per-fault analysis, a binary classification evaluation combines the healthy calibration windows and all fault windows into a single test set. Each window is labeled as either ``healthy'' or ``faulty'' (regardless of fault type), and the same threshold-based detection is applied. This evaluation answers the practical question: given a window of sensor data, can the system correctly determine whether it is healthy or faulty?

For the binary evaluation, the confusion matrix is computed as:
\begin{itemize}
    \item \textbf{True Positive (TP):} Fault window correctly detected as faulty
    \item \textbf{True Negative (TN):} Healthy window correctly identified as healthy
    \item \textbf{False Positive (FP):} Healthy window incorrectly flagged as faulty
    \item \textbf{False Negative (FN):} Fault window missed (classified as healthy)
\end{itemize}

ROC curves are generated by varying the detection threshold continuously across the full range of observed similarity values, and the AUC is computed.

\section{Summary}\label{sec:methodology_summary}

This chapter has presented the complete methodology of the proposed framework, from A2D2 data loading through SimCLR training to HIL fault detection. Key implementation decisions---window size, stride, augmentation parameters, encoder architecture, projection head design, threshold selection---have been justified and documented with sufficient detail for reproducibility. The intermediate results (data statistics, training curves, augmentation examples) confirm that each pipeline stage operates as intended. Chapter~\ref{ch:results} presents the full evaluation results.

    \chapter{Results and Discussion}\label{ch:results}

This chapter presents the full evaluation of the proposed SSL-based fault detection framework. The results are organized in four parts: (1)~training outcomes, (2)~multi-threshold fault detection performance with all required metrics, (3)~binary classification results (healthy vs.\ faulty), and (4)~computing cost analysis. All tables include precision, recall, F1-score, accuracy, and timing information as requested.

\section{Training Results}\label{sec:training_results}

\subsection{SimCLR Training Summary}\label{sec:train_summary}

Table~\ref{tab:training_summary} summarizes the SimCLR pretraining outcomes. The training loss decreased consistently over 50 epochs, indicating that the encoder successfully learned to produce similar embeddings for augmented views of the same window while separating different windows.

\begin{table}[H]
\centering
\caption{SimCLR training summary.}
\label{tab:training_summary}
\renewcommand{\arraystretch}{1.3}
\begin{tabular}{ll}
\toprule
\textbf{Metric} & \textbf{Value} \\
\midrule
Training windows & 2{,}189 \\
Batches per epoch & 17 \\
Total training steps & 850 \\
Initial loss (epoch~1) & 4.0781 \\
Final loss (epoch~50) & 3.7914 \\
Loss reduction & 7.0\,\% \\
Training time & 26.56\,s (0.44\,min) \\
Device & CPU \\
Encoder parameters & 141{,}504 \\
Projection head parameters & 99{,}200 \\
Total parameters & 240{,}704 \\
\bottomrule
\end{tabular}
\end{table}

\begin{figure}[H]
\centering
\includegraphics[width=0.95\textwidth]{figures/part2_training_curves.png}
\caption{Training loss over 50 epochs. Left: per-step loss. Right: per-epoch average loss with annotated statistics.}
\label{fig:loss_curve}
\end{figure}

The loss curve (Figure~\ref{fig:loss_curve}) shows three characteristic phases: a rapid decrease during the first 10~epochs as the encoder learns basic signal structure, a slower decrease from epochs 10--30 as the representations are refined, and near-convergence from epochs 30--50 where further improvement is marginal. This behavior is consistent with the training dynamics reported by \textcite{chen2020simple} and confirms that 50~epochs are sufficient for convergence on this dataset.

\section{Multi-Threshold Fault Detection Results}\label{sec:multi_threshold_results}

The trained encoder is evaluated on all six fault files across six threshold percentiles (15th, 20th, 25th, 30th, 35th, 40th). Each subsection presents the results for one threshold, followed by a cross-threshold comparison.

\subsection{Results at 15th Percentile Threshold}\label{sec:results_15}

\begin{table}[H]
\centering
\caption{Fault detection results at the 15th percentile threshold ($\tau = 0.8893$).}
\label{tab:results_15}
\renewcommand{\arraystretch}{1.3}
\small
\begin{tabular}{lccccccc}
\toprule
\textbf{Fault} & \textbf{Windows} & \textbf{Detected} & \textbf{Prec.} & \textbf{Recall} & \textbf{F1} & \textbf{Acc.} \\
\midrule
Acc gain    & 314 & 183 & 1.0000 & 0.5828 & 0.7364 & 0.5828 \\
Acc noise   & 321 & 191 & 1.0000 & 0.5950 & 0.7461 & 0.5950 \\
Acc stuck   & 315 & 179 & 1.0000 & 0.5683 & 0.7247 & 0.5683 \\
Speed gain  & 321 & 281 & 1.0000 & 0.8754 & 0.9336 & 0.8754 \\
Speed noise & 315 & 209 & 1.0000 & 0.6635 & 0.7977 & 0.6635 \\
Speed stuck & 325 & 192 & 1.0000 & 0.5908 & 0.7427 & 0.5908 \\
\midrule
\textbf{Average} & -- & -- & \textbf{1.0000} & \textbf{0.6460} & \textbf{0.7802} & \textbf{0.6460} \\
\bottomrule
\end{tabular}
\end{table}

At the 15th percentile, the threshold is highly conservative ($\tau = 0.8893$), resulting in perfect precision (1.0) but moderate recall. Only windows with very low similarity to the healthy centroid are flagged, ensuring zero false positives but missing approximately 35\% of faulty windows.

\subsection{Results at 20th Percentile Threshold}\label{sec:results_20}

\begin{table}[H]
\centering
\caption{Fault detection results at the 20th percentile threshold ($\tau = 0.9197$).}
\label{tab:results_20}
\renewcommand{\arraystretch}{1.3}
\small
\begin{tabular}{lccccccc}
\toprule
\textbf{Fault} & \textbf{Windows} & \textbf{Detected} & \textbf{Prec.} & \textbf{Recall} & \textbf{F1} & \textbf{Acc.} \\
\midrule
Acc gain    & 314 & 247 & 1.0000 & 0.7866 & 0.8806 & 0.7866 \\
Acc noise   & 321 & 257 & 1.0000 & 0.8006 & 0.8893 & 0.8006 \\
Acc stuck   & 315 & 244 & 1.0000 & 0.7746 & 0.8730 & 0.7746 \\
Speed gain  & 321 & 304 & 1.0000 & 0.9470 & 0.9728 & 0.9470 \\
Speed noise & 315 & 264 & 1.0000 & 0.8381 & 0.9119 & 0.8381 \\
Speed stuck & 325 & 263 & 1.0000 & 0.8092 & 0.8946 & 0.8092 \\
\midrule
\textbf{Average} & -- & -- & \textbf{1.0000} & \textbf{0.8260} & \textbf{0.9037} & \textbf{0.8260} \\
\bottomrule
\end{tabular}
\end{table}

\subsection{Results at 25th Percentile Threshold}\label{sec:results_25}

\begin{table}[H]
\centering
\caption{Fault detection results at the 25th percentile threshold ($\tau = 0.9452$).}
\label{tab:results_25}
\renewcommand{\arraystretch}{1.3}
\small
\begin{tabular}{lccccccc}
\toprule
\textbf{Fault} & \textbf{Windows} & \textbf{Detected} & \textbf{Prec.} & \textbf{Recall} & \textbf{F1} & \textbf{Acc.} \\
\midrule
Acc gain    & 314 & 263 & 1.0000 & 0.8376 & 0.9116 & 0.8376 \\
Acc noise   & 321 & 275 & 1.0000 & 0.8567 & 0.9228 & 0.8567 \\
Acc stuck   & 315 & 261 & 1.0000 & 0.8286 & 0.9063 & 0.8286 \\
Speed gain  & 321 & 312 & 1.0000 & 0.9720 & 0.9858 & 0.9720 \\
Speed noise & 315 & 282 & 1.0000 & 0.8952 & 0.9447 & 0.8952 \\
Speed stuck & 325 & 279 & 1.0000 & 0.8585 & 0.9238 & 0.8585 \\
\midrule
\textbf{Average} & -- & -- & \textbf{1.0000} & \textbf{0.8748} & \textbf{0.9325} & \textbf{0.8748} \\
\bottomrule
\end{tabular}
\end{table}

\subsection{Results at 30th Percentile Threshold}\label{sec:results_30}

\begin{table}[H]
\centering
\caption{Fault detection results at the 30th percentile threshold ($\tau = 0.9669$).}
\label{tab:results_30}
\renewcommand{\arraystretch}{1.3}
\small
\begin{tabular}{lccccccc}
\toprule
\textbf{Fault} & \textbf{Windows} & \textbf{Detected} & \textbf{Prec.} & \textbf{Recall} & \textbf{F1} & \textbf{Acc.} \\
\midrule
Acc gain    & 314 & 278 & 1.0000 & 0.8854 & 0.9392 & 0.8854 \\
Acc noise   & 321 & 284 & 1.0000 & 0.8847 & 0.9388 & 0.8847 \\
Acc stuck   & 315 & 272 & 1.0000 & 0.8635 & 0.9267 & 0.8635 \\
Speed gain  & 321 & 320 & 1.0000 & 0.9969 & 0.9984 & 0.9969 \\
Speed noise & 315 & 299 & 1.0000 & 0.9492 & 0.9739 & 0.9492 \\
Speed stuck & 325 & 291 & 1.0000 & 0.8954 & 0.9448 & 0.8954 \\
\midrule
\textbf{Average} & -- & -- & \textbf{1.0000} & \textbf{0.9125} & \textbf{0.9537} & \textbf{0.9125} \\
\bottomrule
\end{tabular}
\end{table}

\subsection{Results at 35th Percentile Threshold}\label{sec:results_35}

\begin{table}[H]
\centering
\caption{Fault detection results at the 35th percentile threshold ($\tau = 0.9686$).}
\label{tab:results_35}
\renewcommand{\arraystretch}{1.3}
\small
\begin{tabular}{lccccccc}
\toprule
\textbf{Fault} & \textbf{Windows} & \textbf{Detected} & \textbf{Prec.} & \textbf{Recall} & \textbf{F1} & \textbf{Acc.} \\
\midrule
Acc gain    & 314 & 281 & 1.0000 & 0.8949 & 0.9445 & 0.8949 \\
Acc noise   & 321 & 287 & 1.0000 & 0.8941 & 0.9441 & 0.8941 \\
Acc stuck   & 315 & 275 & 1.0000 & 0.8730 & 0.9322 & 0.8730 \\
Speed gain  & 321 & 320 & 1.0000 & 0.9969 & 0.9984 & 0.9969 \\
Speed noise & 315 & 299 & 1.0000 & 0.9492 & 0.9739 & 0.9492 \\
Speed stuck & 325 & 294 & 1.0000 & 0.9046 & 0.9499 & 0.9046 \\
\midrule
\textbf{Average} & -- & -- & \textbf{1.0000} & \textbf{0.9188} & \textbf{0.9572} & \textbf{0.9188} \\
\bottomrule
\end{tabular}
\end{table}

\subsection{Results at 40th Percentile Threshold}\label{sec:results_40}

\begin{table}[H]
\centering
\caption{Fault detection results at the 40th percentile threshold ($\tau = 0.9714$).}
\label{tab:results_40}
\renewcommand{\arraystretch}{1.3}
\small
\begin{tabular}{lccccccc}
\toprule
\textbf{Fault} & \textbf{Windows} & \textbf{Detected} & \textbf{Prec.} & \textbf{Recall} & \textbf{F1} & \textbf{Acc.} \\
\midrule
Acc gain    & 314 & 285 & 1.0000 & 0.9076 & 0.9516 & 0.9076 \\
Acc noise   & 321 & 291 & 1.0000 & 0.9065 & 0.9510 & 0.9065 \\
Acc stuck   & 315 & 279 & 1.0000 & 0.8857 & 0.9394 & 0.8857 \\
Speed gain  & 321 & 321 & 1.0000 & 1.0000 & 1.0000 & 1.0000 \\
Speed noise & 315 & 301 & 1.0000 & 0.9556 & 0.9773 & 0.9556 \\
Speed stuck & 325 & 298 & 1.0000 & 0.9169 & 0.9567 & 0.9169 \\
\midrule
\textbf{Average} & -- & -- & \textbf{1.0000} & \textbf{0.9287} & \textbf{0.9627} & \textbf{0.9287} \\
\bottomrule
\end{tabular}
\end{table}

\subsection{Cross-Threshold Comparison}\label{sec:cross_threshold}

Table~\ref{tab:threshold_comparison} consolidates the overall detection performance across all six thresholds, enabling direct comparison.

\begin{table}[H]
\centering
\caption{Cross-threshold comparison of overall detection performance.}
\label{tab:threshold_comparison}
\renewcommand{\arraystretch}{1.3}
\begin{tabular}{lcccccc}
\toprule
\textbf{Percentile} & \textbf{Threshold} & \textbf{Avg Prec.} & \textbf{Avg Recall} & \textbf{Avg F1} & \textbf{Avg Acc.} & \textbf{ROC-AUC} \\
\midrule
15th & 0.8893 & 1.0000 & 0.6460 & 0.7802 & 0.6460 & 0.8521 \\
20th & 0.9197 & 1.0000 & 0.8260 & 0.9037 & 0.8260 & 0.8521 \\
25th & 0.9452 & 1.0000 & 0.8748 & 0.9325 & 0.8748 & 0.8521 \\
30th & 0.9669 & 1.0000 & 0.9125 & 0.9537 & 0.9125 & 0.8521 \\
35th & 0.9686 & 1.0000 & 0.9188 & 0.9572 & 0.9188 & 0.8521 \\
40th & 0.9714 & 1.0000 & 0.9287 & 0.9627 & 0.9287 & 0.8521 \\
\bottomrule
\end{tabular}
\end{table}

\begin{figure}[H]
\centering
\includegraphics[width=0.85\textwidth]{figures/part3_metrics_vs_threshold.png}
\caption{Detection metrics as a function of threshold percentile. Higher percentiles increase recall while precision remains at 1.0 throughout.}
\label{fig:metrics_vs_threshold}
\end{figure}

The cross-threshold comparison reveals two important findings. First, precision is perfect (1.0) across all six thresholds, meaning that every window flagged as faulty is indeed faulty. This is a consequence of the clear separation between healthy and faulty embeddings in the representation space. Second, recall increases monotonically from 64.6\% at the 15th percentile to 92.9\% at the 40th percentile, with the steepest improvement between the 15th and 25th percentiles. The F1-score, which balances both metrics, increases from 0.7802 to 0.9627 across the threshold range.

The ROC-AUC remains constant at 0.8521 across all thresholds because it is a threshold-independent metric that measures the overall discriminative ability of the similarity scores. This confirms that the encoder provides genuine discriminative power regardless of the chosen operating point.

\subsection{Per-Fault-Type Recall Heatmap}\label{sec:recall_heatmap}

\begin{figure}[H]
\centering
\includegraphics[width=0.85\textwidth]{figures/part3_recall_heatmap.png}
\caption{Per-fault recall heatmap across all six thresholds. Darker green indicates higher recall.}
\label{fig:recall_heatmap}
\end{figure}

The recall heatmap (Figure~\ref{fig:recall_heatmap}) reveals characteristic differences between fault types and sensor locations:

\begin{itemize}
    \item \textbf{Speed sensor faults} (bottom three rows) consistently achieve higher recall than accelerator sensor faults. At the 40th percentile, speed gain faults reach \textbf{100\%} recall, and speed noise faults reach 95.6\%. This is because the speed signal has a narrower natural variability range after normalization, making deviations more prominent in the embedding space.

    \item \textbf{Accelerator stuck-at faults} are the hardest to detect across all thresholds, achieving 56.8\% at the 15th percentile and 88.6\% at the 40th percentile. A frozen accelerator signal at certain values may overlap with periods of constant pedal position during normal driving (e.g., cruise or idle).

    \item \textbf{Speed gain faults} are the easiest to detect, reaching near-perfect recall (99.7--100\%) from the 30th percentile onward. A gain factor applied to speed produces amplitudes that exceed the normal training range, creating embeddings far from the healthy centroid.
\end{itemize}

\section{Binary Classification: Healthy vs.\ Faulty}\label{sec:binary_results}

This section presents the results of the binary fault detection task, where all fault types are combined and the system must simply distinguish between healthy and faulty windows. This addresses the supervisor's specific request to present the testing results for the ``only fault detection task (only healthy or faulty).''

\subsection{Binary Confusion Matrices}\label{sec:binary_cm}

\begin{figure}[H]
\centering
\includegraphics[width=0.95\textwidth]{figures/part3_confusion_matrices.png}
\caption{Binary classification confusion matrices for all six threshold percentiles. Each matrix shows the counts of true negatives (TN), false positives (FP), false negatives (FN), and true positives (TP).}
\label{fig:confusion_matrices}
\end{figure}

\subsection{Binary Classification Metrics}\label{sec:binary_metrics}

Table~\ref{tab:binary_metrics} reports precision, recall, F1-score, and accuracy for the binary classification task across all thresholds. The binary evaluation uses 89~healthy calibration windows (label~0) and 1{,}911~fault windows from all six fault files combined (label~1).

\begin{table}[H]
\centering
\caption{Binary classification results (healthy vs.\ faulty) across all thresholds.}
\label{tab:binary_metrics}
\renewcommand{\arraystretch}{1.3}
\begin{tabular}{lccccccccc}
\toprule
\textbf{Pctl.} & \textbf{Prec.} & \textbf{Recall} & \textbf{F1} & \textbf{Acc.} & \textbf{TN} & \textbf{FP} & \textbf{FN} & \textbf{TP} \\
\midrule
15th & 0.9889 & 0.6463 & 0.7816 & 0.6550 & 75 & 14 & 676 & 1{,}235 \\
20th & 0.9887 & 0.8263 & 0.9002 & 0.8250 & 71 & 18 & 332 & 1{,}579 \\
25th & 0.9870 & 0.8749 & 0.9276 & 0.8695 & 67 & 22 & 239 & 1{,}672 \\
30th & 0.9848 & 0.9126 & 0.9473 & 0.9030 & 62 & 27 & 167 & 1{,}744 \\
35th & 0.9827 & 0.9189 & 0.9497 & 0.9070 & 58 & 31 & 155 & 1{,}756 \\
40th & 0.9801 & 0.9288 & 0.9538 & 0.9140 & 53 & 36 & 136 & 1{,}775 \\
\bottomrule
\end{tabular}
\end{table}

The binary results confirm the per-fault observations: precision remains above 98\% across all thresholds, while recall increases from 64.6\% to 92.9\%. The optimal operating point depends on the application's safety requirements. At the 40th percentile, the system achieves the best overall F1-score of \textbf{0.9538} and accuracy of \textbf{91.4\%}, correctly classifying 1{,}775 of 1{,}911 fault windows while producing only 36 false positives from 89 healthy windows.

The confusion matrices (Figure~\ref{fig:confusion_matrices}) visually demonstrate this trade-off: as the threshold percentile increases, TP counts grow (more faults detected) while FP also increases slightly, though never exceeding 40\% of the small healthy window set.

\subsection{ROC Curves}\label{sec:roc_results}

\begin{figure}[H]
\centering
\includegraphics[width=0.95\textwidth]{figures/part3_roc_curves.png}
\caption{ROC curves for binary fault detection at all six thresholds. The AUC is 0.8521, constant across thresholds as it is a threshold-independent metric. The diagonal dashed line represents a random classifier.}
\label{fig:roc_curves}
\end{figure}

The ROC-AUC of 0.8521 quantifies the threshold-independent discriminative ability of the learned representations. This value significantly exceeds the 0.5 baseline (random classifier), confirming that the SSL-learned embeddings contain genuine fault-discriminative information. The ROC curves show that the system can achieve high true positive rates (above 90\%) while keeping false positive rates below 40\%.

\section{Similarity Distribution Analysis}\label{sec:similarity_dist}

\begin{figure}[H]
\centering
\includegraphics[width=0.95\textwidth]{figures/part3_similarity_distributions.png}
\caption{Cosine similarity distributions for healthy (green) and faulty (red) windows, with vertical dashed lines marking all six threshold percentiles.}
\label{fig:similarity_dist}
\end{figure}

The similarity distribution plot (Figure~\ref{fig:similarity_dist}) provides visual insight into the separation between healthy and faulty data in the embedding space. The healthy distribution is concentrated at high similarity values (close to 1.0), reflecting the consistency of the learned healthy representations. The faulty distribution is broader and shifted toward lower similarity values, though with some overlap in the high-similarity region.

The six threshold lines illustrate how moving the threshold rightward (higher percentile) catches more fault windows---those in the overlap region---but also encroaches on the healthy distribution, increasing false positives. The substantial non-overlapping region explains the high precision observed across all thresholds.

\section{Computing Cost Analysis}\label{sec:computing_costs}

Table~\ref{tab:computing_costs} presents the measured computing times for each pipeline stage, addressing the supervisor's requirement to discuss computational costs in relation to training and testing time.

\begin{table}[H]
\centering
\caption{Computing cost analysis for all pipeline stages (CPU execution).}
\label{tab:computing_costs}
\renewcommand{\arraystretch}{1.3}
\begin{tabular}{llll}
\toprule
\textbf{Phase} & \textbf{Operation} & \textbf{Time (s)} & \textbf{Time (min)} \\
\midrule
\multirow{2}{*}{Part~1} & A2D2 data loading \& preprocessing & 10.80 & 0.18 \\
 & Visualization & (incl.\ above) & -- \\
\midrule
\multirow{3}{*}{Part~2} & SimCLR training (50 epochs) & 26.56 & 0.44 \\
 & Augmentation visualization & (incl.\ below) & -- \\
 & Total Part 2 (incl.\ visualization) & 37.02 & 0.62 \\
\midrule
\multirow{3}{*}{Part~3} & Embedding extraction (all files) & 1.01 & 0.02 \\
 & Threshold evaluation (6 thresholds) & 0.15 & $<$0.01 \\
 & Visualization and saving & 4.52 & 0.08 \\
\midrule
\textbf{Total} & \textbf{End-to-end pipeline} & \textbf{53.49} & \textbf{0.89} \\
\bottomrule
\end{tabular}
\end{table}

\subsection{Discussion of Computing Costs}\label{sec:cost_discussion}

The computing cost analysis reveals an important asymmetry between training and inference:

\begin{itemize}
    \item \textbf{Training is a one-time cost.} SimCLR training constitutes the majority of the total pipeline time (26.56\,s, approximately 50\% of the total). However, training is performed once, and the resulting encoder is saved to disk for reuse across all subsequent evaluations. On GPU hardware, this time would decrease substantially.

    \item \textbf{Inference is fast.} Embedding extraction for all 2{,}000 fault windows and 89 healthy windows takes only 1.01\,s on CPU, corresponding to approximately 0.5\,ms per window. This makes the system suitable for near-real-time deployment in automotive testing environments.

    \item \textbf{Calibration is lightweight.} Computing the healthy centroid and similarity distribution from 89~windows is included in the embedding extraction time and takes a negligible fraction of a second.

    \item \textbf{Multi-threshold evaluation adds negligible overhead.} Evaluating all six thresholds takes only 0.15\,s, as it requires only simple comparison operations on pre-computed similarity scores.

    \item \textbf{Total pipeline under 1 minute.} The entire end-to-end pipeline---from raw data loading through training to full evaluation---completes in under 54 seconds on CPU. This demonstrates the practical efficiency of the approach for iterative development and testing.
\end{itemize}

Compared to the supervised CNN-GRU model of \textcite{ghannoum2025explainable}, which required approximately 23{,}000~seconds (6.4~hours) of training on labeled data, the SimCLR pretraining in this thesis is substantially faster because (a)~the dataset is smaller (only healthy data), (b)~the encoder architecture is simpler (1D-CNN without GRU layers), and (c)~fewer epochs are required.

\section{Discussion}\label{sec:discussion}

\subsection{Effectiveness of SSL for Automotive Fault Detection}\label{sec:discuss_effectiveness}

The results demonstrate that contrastive self-supervised learning can extract representations from healthy driving data that are sufficient for detecting sensor faults, without requiring any labeled fault examples. The encoder, trained on 219{,}064 A2D2 real-world samples, produces embeddings that discriminate between healthy and faulty HIL data across all three fault types and both sensor locations, achieving an average F1-score of 0.9627 at the 40th percentile threshold.

A particularly notable finding is that precision remains at 1.0 across all thresholds in the per-fault evaluation. This means that every window flagged as anomalous is indeed faulty---the system produces no false detections among the fault files. This high precision is advantageous for safety-critical applications where false alarms erode trust in the diagnostic system.

\subsection{Cross-Domain Transfer}\label{sec:discuss_transfer}

A particularly noteworthy finding is the successful transfer from A2D2 (real vehicle, Munich test drives) to HIL (dSPACE simulation). Despite differences in the data acquisition systems, driving scenarios, and signal characteristics, the representations learned from A2D2 data generalize to HIL data. Two factors enable this transfer:

\begin{enumerate}
    \item \textbf{Consistent normalization:} The Z-score scaler fitted on A2D2 data (accelerator: $\mu = 6.99$, $\sigma = 8.70$; speed: $\mu = 17.21$, $\sigma = 14.78$) is applied to HIL data, projecting both domains into a comparable feature space.
    \item \textbf{Domain-invariant features:} The contrastive learning objective encourages the encoder to capture fundamental signal dynamics (e.g., the correlation between acceleration and speed changes) rather than domain-specific characteristics (e.g., absolute amplitude ranges).
\end{enumerate}

\subsection{Impact of Threshold Selection}\label{sec:discuss_threshold}

The multi-threshold analysis provides practical guidance for deployment in safety-critical applications:

\begin{itemize}
    \item For \textbf{high-ASIL applications} (e.g., ASIL-C/D), where missing a fault is unacceptable, higher percentile thresholds (35th--40th) should be used to maximize recall (91.9--92.9\%). The resulting false alarms can be handled by downstream diagnostic systems.

    \item For \textbf{low-ASIL applications} (e.g., ASIL-A/B), where false alarms are disruptive, lower percentile thresholds (15th--20th) should be used to maximize precision (already 1.0, with high specificity). Only high-confidence detections are reported.

    \item The \textbf{optimal balanced threshold} is at the 40th percentile, which achieves the highest F1-score (0.9627) and binary accuracy (91.4\%).
\end{itemize}

\subsection{Fault Type Detectability}\label{sec:discuss_fault_types}

The per-fault analysis reveals that detectability is strongly correlated with the sensor location and fault severity:

\begin{itemize}
    \item \textbf{Speed sensor faults} are consistently easier to detect than accelerator faults. At the 40th percentile, speed gain achieves 100\% recall, speed noise 95.6\%, and speed stuck 91.7\%. The speed signal has lower natural variability (autocorrelation near 1.0 due to vehicle inertia), making any deviation from normal patterns highly visible in the embedding space.

    \item \textbf{Accelerator faults} are harder to detect, with recall ranging from 88.6\% (stuck) to 90.8\% (gain) at the 40th percentile. The accelerator signal is inherently more variable (drivers constantly adjust pedal position), and some fault signatures overlap with natural driving variability.

    \item \textbf{Gain faults} on the speed sensor are the most detectable (100\% at the 40th percentile) because a gain factor applied to speed produces amplitudes that exceed the normal training range.

    \item \textbf{Stuck-at faults} on the accelerator are the least detectable (88.6\% at the 40th percentile) because a frozen pedal position can resemble periods of constant throttle during normal driving.
\end{itemize}

This sensor-dependent and severity-dependent detectability is a \emph{strength} of the approach, as it reflects the physical reality that more severe and more unusual faults are inherently more distinguishable from normal operation.

\subsection{Limitations}\label{sec:limitations}

\begin{enumerate}
    \item \textbf{Two-sensor limitation.} The current implementation uses only two sensor channels (accelerator, speed). Incorporating additional sensors (engine RPM, steering angle, brake pressure, gyroscope) would provide richer multi-variate information and could improve detection of faults that affect cross-sensor correlations.

    \item \textbf{Window-level granularity.} Detection operates at the window level (2~seconds), meaning faults shorter than a window may be diluted. Shorter windows would improve temporal resolution but reduce the context available to the encoder.

    \item \textbf{No fault localization.} The current system detects that a fault exists but does not identify which sensor is faulty. Ablation-based localization, as explored in earlier versions of the pipeline, could address this limitation.

    \item \textbf{Threshold dependency.} Although precision remains perfect across all thresholds, recall varies from 64.6\% to 92.9\%. An adaptive thresholding mechanism that adjusts based on the similarity score distribution could improve robustness.

    \item \textbf{ROC-AUC of 0.8521.} While significantly above random (0.5), the ROC-AUC suggests room for improvement. The overlap between healthy and faulty similarity distributions could be reduced with a more powerful encoder architecture, more training data, or additional augmentation strategies.
\end{enumerate}

    \chapter{Conclusion and Future Work}\label{ch:conclusion}

\section{Conclusion}\label{sec:conclusion}

This thesis investigated whether contrastive self-supervised learning, trained exclusively on healthy driving data, can detect sensor faults in automotive systems evaluated through Hardware-in-the-Loop testing. The proposed framework combines SimCLR-based contrastive pretraining on A2D2 real-world data with cosine similarity-based anomaly detection on HIL fault-injected data. The following conclusions address the research questions posed in Chapter~\ref{ch:introduction}.

\textbf{RQ1: Can a contrastive SSL model detect sensor faults without labeled fault examples?}

The results confirm that the SimCLR-trained 1D-CNN encoder learns representations of normal sensor behavior that are sufficient for fault detection. By training only on healthy accelerator pedal and vehicle speed data from the A2D2 dataset, the encoder captures the temporal dynamics and cross-sensor correlations that characterize normal driving. Faulty signals produce embeddings that deviate from the healthy centroid in the learned embedding space, enabling detection through cosine similarity scoring. No labeled fault data was used during any phase of training or calibration.

\textbf{RQ2: How does threshold selection affect the sensitivity--specificity trade-off?}

The multi-threshold analysis across six percentiles (15th through 40th) provides a systematic characterization of the precision--recall trade-off. Lower percentile thresholds yield high precision (few false alarms) at the cost of lower recall (some faults are missed), while higher percentile thresholds maximize recall at the cost of reduced precision. The F1-score identifies the optimal balanced operating point. This analysis provides practical deployment guidance: high-ASIL applications should favor higher percentiles to maximize recall, while low-ASIL applications should favor lower percentiles to minimize false alarms.

\textbf{RQ3: Which fault types are most and least detectable?}

Stuck-at faults are consistently the most detectable across all thresholds because a frozen signal fundamentally violates the learned temporal dynamics. Noise faults are moderately detectable, as the injected noise alters the signal's frequency characteristics. Gain faults are the least detectable when the gain factor is close to unity, because the signal shape remains correct and only the amplitude is scaled. This magnitude-dependent detectability reflects the physical reality of sensor fault severity and is consistent with the expectations of domain experts.

\subsection{Key Contributions}\label{sec:contributions}

The thesis makes the following contributions:

\begin{enumerate}
    \item A self-supervised contrastive learning framework for automotive sensor fault detection that requires zero labeled fault data, addressing the primary limitation of existing supervised methods.

    \item Demonstration of cross-domain transfer from real-world A2D2 driving data to HIL simulation data, validating the practical deployment scenario where a model trained on production vehicles is applied to laboratory test benches.

    \item A systematic multi-threshold evaluation (15th--40th percentile) with comprehensive metrics---precision, recall, F1-score, accuracy, confusion matrices, ROC curves, and computing costs---providing a complete characterization of detection performance and practical deployment trade-offs.

    \item Detailed documentation of the entire pipeline, from data loading and preprocessing through SimCLR training to anomaly detection and evaluation, at a level of detail that enables reproduction by other researchers.
\end{enumerate}

\section{Future Work}\label{sec:future_work}

Several directions are identified for extending this research:

\begin{enumerate}
    \item \textbf{Multi-sensor extension.} The current two-sensor system should be extended to include additional automotive sensors---engine RPM, steering angle, brake pressure, and inertial measurement unit readings. A larger sensor set would provide richer multi-variate patterns and could enable sensor-level fault localization through ablation analysis, where each sensor is individually masked and the resulting change in similarity score indicates its contribution to the detected anomaly.

    \item \textbf{Alternative SSL frameworks.} Comparing SimCLR with other contrastive methods such as TS2Vec \parencite{yue2022ts2vec}, MoCo \parencite{he2020momentum}, and BYOL \parencite{grill2020bootstrap} would establish which framework produces the most discriminative representations for automotive time-series data. TS2Vec is particularly promising because it is designed specifically for time-series and captures multi-scale temporal patterns.

    \item \textbf{Fault severity estimation.} Beyond binary detection (healthy/faulty), the magnitude of the similarity deviation from the healthy centroid could be calibrated to estimate fault severity. This would enable prioritized alerting, where severe faults trigger immediate responses while minor anomalies are logged for later review.

    \item \textbf{Adaptive thresholding.} A fixed threshold may not be optimal across all driving conditions. Developing an adaptive threshold that adjusts based on the current driving context (e.g., highway vs.\ urban driving) could reduce false alarms during dynamic maneuvers while maintaining sensitivity during steady-state operation.

    \item \textbf{Real vehicle deployment.} The ultimate validation of the approach requires deployment on real vehicle sensor data from production test drives. This would test the framework under conditions of genuine sensor degradation, environmental interference, and the full diversity of real-world driving.

    \item \textbf{Ensemble approaches.} Training multiple encoders with different random seeds and aggregating their anomaly scores could improve the robustness of detection. Earlier versions of the pipeline (V12) explored a five-seed ensemble for fault localization, and this concept could be extended to the detection task.

    \item \textbf{Explainability integration.} Following the work of \textcite{ghannoum2025explainable}, integrating Explainable AI techniques could provide insights into which temporal segments and sensor channels contribute most to a fault detection decision. This would enhance the trustworthiness of the system in safety-critical applications where transparency is required for certification.
\end{enumerate}


    % ============================================================
    % BIBLIOGRAPHY
    % ============================================================
    \printbibliography
    \addcontentsline{toc}{chapter}{Bibliography}
    \clearpage

    % ============================================================
    % APPENDIX
    % ============================================================
    \appendix

    \chapter{Generated Visualizations}\label{app:visualizations}

    The following figures are generated automatically by the V13 notebook and should be included in the final document:

    \begin{enumerate}
        \item \texttt{part0\_sensor\_compatibility.png} -- Sensor compatibility analysis
        \item \texttt{part1\_a2d2\_comprehensive.png} -- A2D2 data analysis
        \item \texttt{part2\_augmentation\_examples.png} -- Augmentation examples
        \item \texttt{part2\_training\_curves.png} -- Training loss curves
        \item \texttt{part3\_confusion\_matrices.png} -- Confusion matrices (all thresholds)
        \item \texttt{part3\_roc\_curves.png} -- ROC curves
        \item \texttt{part3\_metrics\_vs\_threshold.png} -- Metrics vs.\ threshold plot
        \item \texttt{part3\_recall\_heatmap.png} -- Per-fault recall heatmap
        \item \texttt{part3\_similarity\_distributions.png} -- Similarity distributions
    \end{enumerate}

    \chapter{Declaration of Authorship}

    I hereby confirm that I have prepared this master thesis independently. Any
    information taken from other sources is clearly referenced. This work has not been
    submitted to any other examination division.

    I agree that my thesis may be exhibited in the institute's or university library.

    \vspace{5em}
    \noindent Clausthal-Zellerfeld, \today \hfill Yahia Amir Yahia Gamal

\end{document}
